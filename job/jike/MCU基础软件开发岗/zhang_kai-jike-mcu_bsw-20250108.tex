% !TEX program = xelatex
\documentclass[11pt,a4paper]{ctexart}

% ============ 基础包 ============
\usepackage[margin=2cm]{geometry}
\usepackage{ctex}
\usepackage{fancyhdr}
\usepackage{titlesec}
\usepackage{enumitem}
\usepackage{graphicx}
\usepackage{hyperref}
\usepackage{xcolor}
\usepackage{tabularx}
\usepackage{array}
\usepackage{multirow}
\usepackage{pifont}
\usepackage{wrapfig}
\usepackage{float}

% ============ 页面设置 ============
\geometry{left=1.8cm, right=1.8cm, top=2cm, bottom=2cm}
\pagestyle{fancy}
\fancyhf{}
\fancyfoot[C]{\thepage}
\renewcommand{\headrulewidth}{0pt}

% ============ 颜色定义 ============
\definecolor{primary}{RGB}{0,82,147}      % 汽车蓝
\definecolor{secondary}{RGB}{102,102,102} % 灰色
\definecolor{accent}{RGB}{220,20,60}      % 强调红
\definecolor{techblue}{RGB}{0,114,188}    % 技术蓝
\definecolor{lightgray}{RGB}{240,240,240}

% ============ 字体设置(系统字体) ============
\setCJKmainfont{SimHei}
\setCJKsansfont{SimHei}
\setCJKmonofont{FangSong}

% ============ 标题格式 ============
\titleformat{\section}
  {\Large\bfseries\sffamily\color{primary}}
  {}
  {0em}
  {}
  [{\titlerule[0.5pt]\color{primary!30!black}}]

\titleformat{\subsection}
  {\large\bfseries\sffamily\color{black}}
  {}
  {0em}
  {}

\titlespacing{\section}{0pt}{12pt}{8pt}
\titlespacing{\subsection}{0pt}{8pt}{6pt}

% ============ 列表设置 ============
\setlist{
  leftmargin=0pt,
  itemsep=3pt,
  parsep=0pt,
  topsep=3pt,
  labelsep=8pt
}

% ============ 自定义命令 ============
\newcommand{\highlight}[1]{\textcolor{primary}{\textbf{#1}}}
\newcommand{\tech}[1]{\textcolor{techblue}{\texttt{#1}}}
\newcommand{\skillitem}[2]{\item[\ding{108}] \textbf{#1}: #2}

% ============ 超链接设置 ============
\hypersetup{
  colorlinks=true,
  urlcolor=techblue,
  linkcolor=primary,
  citecolor=secondary
}

\begin{document}

% ============ 个人信息 ============
\begin{minipage}{0.65\textwidth}
  {\Huge\bfseries\sffamily 张凯} \\[8pt]
  {\large\texttt{电话:+86 15122986177} | \texttt{邮箱:\href{mailto:15122986177@163.com}{15122986177@163.com}}} \\[6pt]
  {\large\texttt{上海 | 婚姻状况:已婚 | 出生日期:1992年3月11日}}
\end{minipage}
\hfill
\begin{minipage}{0.3\textwidth}
  \centering
  \includegraphics[width=0.9\textwidth,height=3.5cm,keepaspectratio]{profile_photo.jpg}
\end{minipage}

\vspace{10pt}
{\color{primary}\hrule}
\vspace{12pt}

% ============ 职业概述 ============
\section*{职业概述}

\normalsize
5年\tech{AUTOSAR}软件开发经验,负责过车身域控制器接口架构,具备\tech{DaVinci Developer}中SWC层所有相关配置经验,拥有Davinci Configurator中\tech{SOME/IP}服务相关模块配置经验(\tech{SoAd}、\tech{SD}、\tech{PduR}等),深入了解AUTOSAR CP通讯栈,具备车身域控制器和中央网关计算平台集成开发经验。

% ============ 关键能力 ============
\section*{关键能力}

\begin{itemize}
  \skillitem{Vector工具链}{熟练使用\tech{Vector DaVinci Developer}、\tech{DaVinci Configurator}进行\tech{AUTOSAR}相关配置和代码生成,解决\tech{ARXML}配置冲突和兼容性问题}
  \skillitem{SOME/IP配置}{负责车载以太网\tech{SOME/IP}相关模块配置,包括\tech{SoAd}传输层、\tech{SD}服务发现、\tech{PduR}路由和\tech{TCP/UDP}配置}
  \skillitem{OS时序挂载与优化}{熟悉\tech{AUTOSAR OS}时序挂载和任务调度机制,能够根据系统需求优化任务优先级和调度策略,提升系统响应速度和稳定性}
  \skillitem{调试与问题定位}{熟练使用\tech{Lauterbach Trace32}、\tech{iSYSTEM winIDEA}进行\tech{C}代码级调试,具备量产软件问题定位和系统级问题排查能力}
  \skillitem{自动化测试开发}{基于\tech{Python}开发自动化测试工具,根据客户\tech{CAN/LIN/IO}矩阵自动生成\tech{CANoe CAPL}脚本和\tech{vTESTstudio}配置}
\end{itemize}

\subsection*{技术技能矩阵}

\begin{tabularx}{\textwidth}{@{}p{3.2cm}X@{}}
  \textbf{Vector工具链} & \tech{DaVinci Developer}, \tech{DaVinci Configurator} \\
  \textbf{调试工具} & \tech{Lauterbach Trace32}, \tech{iSYSTEM winIDEA}, \tech{CANoe}
\end{tabularx}

% ============ 工作经历 ============
\section*{工作经历}

\subsection*{沃尔沃汽车(亚太)投资控股有限公司 Volvo Car (Asia Pacific) Investment Holding Co., Ltd.}
\textbf{中央计算平台集成主管工程师} \hfill \textit{2023年11月 -- 至今,中国上海}

\subsubsection*{SPA3/GPA中央计算平台集成|AUTOSAR SOME/IP 开发工程师}
\hfill \textit{2023年11月 -- 2025年7月}

\begin{itemize}
  \item[\ding{51}] \highlight{集成工具开发}:负责\tech{SPA3/GPA}项目工具开发,使用\tech{Python}开发\tech{yaml2arxml}管理\tech{SWC}配置,生成\tech{ECUExtract}和代码
  \item[\ding{51}] \highlight{配置验证与消错}:负责\tech{ECUExtract.arxml}配置检查,完成\tech{Davinci Developer}和\tech{Configurator}配置消错,解决配置冲突问题
  \item[\ding{51}] \highlight{SOME/IP配置与测试}:负责\tech{SOME/IP}服务配置(\tech{SoAd}、\tech{SD}、\tech{TcpIp}),使用\tech{CANoe}配合\tech{VN5650}测试,使用\tech{劳德巴赫}调试\tech{C代码}
\end{itemize}

\subsubsection*{CI/CD架构与工具链开发|流水线设计开发者}
\hfill \textit{2025年8月 -- 至今}

\begin{itemize}
  \item[\ding{51}] \highlight{CI/CD流水线架构与脚本开发}:设计企业级\tech{CI/CD}流水线系统,基于\tech{Jenkins-Gerrit}技术栈实现代码质量门禁系统,开发\tech{Groovy}脚本并集成静态检查工具
  \item[\ding{51}] \highlight{并发架构设计}:设计主从\tech{Job}并发执行架构,使用多进程并发,提升\tech{CI/CD}流水线执行效率\textbf{30\%}
  \item[\ding{51}] \highlight{可配置化开发}:设计\tech{YAML}配置驱动的测试框架,支持同一个\tech{Jenkins Job}的多版本\tech{pipeline}脚本的动态加载和执行
\end{itemize}

\subsection*{科世达(上海)机电有限公司 KOSTAL (Shanghai) Mechatronic Co., Ltd.}
\textbf{车身域控制器开发工程师} \hfill \textit{2020年12月 -- 2023年11月,中国上海}

\subsubsection*{车身域控制器集成负责人}
\hfill \textit{2020年12月 -- 2023年11月}

\begin{itemize}
  \item[\ding{51}] \highlight{车身域控制器软件架构}: 进行\tech{AUTOSAR SWC}架构设计和\tech{Davinci Developer}中所有相关配置,在满足各\tech{SWC}接口需求的前提下将\tech{SWC}开发与\tech{BSW}完全解耦
  \item[\ding{51}] \highlight{车身域控制器BSW存储模块开发}: 负责\tech{NVM}模块开发,实现\tech{SWC}配置数据、诊断(\tech{DID}、\tech{RID}、\tech{DTC})等数据的持久化存储
  \item[\ding{51}] \highlight{车身域控制器集成测试}:搭建车身域控制器的集成测试硬件和软件环境,负责软件编译、烧录和集成测试,定位集成测试和系统测试中的问题并与相关工程师沟通改进软件
  \item[\ding{51}] \highlight{自动化测试工具开发}:基于\tech{Python}开发自动化测试脚本生成工具,根据客户\tech{CAN/LIN/IO}通信矩阵自动生成\tech{CANoe CAPL}脚本和\tech{vTESTstudio}配置文件,在\tech{vTESTstudio}中构建自动化测试项目并编写自动化测试用例
\end{itemize}

\subsubsection*{客户项目经验}
\hfill \textit{2021年1月 -- 2023年8月}

\begin{itemize}
  \item[\ding{108}] \textbf{长城汽车} - 欧拉好猫/黑猫系列\tech{KBCM}项目|集成开发负责人
  \begin{itemize}
    \item 进行AUTOSAR架构设计和\tech{RTE}配置, 进行\tech{OS}时序挂载和任务调度优化
    \item 负责车身域控制器的集成测试和问题定位,开发\tech{BLE, TBOX, VCU, ESCL}认证模块, 开发\tech{NVM Manager}模块和\tech{IOHWab}模块
    \item 负责量产软件发布和客户支持,协调相关工程师解决客户问题
  \end{itemize}

  \item[\ding{108}] \textbf{宇通客车} - \tech{KBCM}车身控制器项目|软件开发工程师
  \begin{itemize}
    \item 负责商用车车身控制器的内灯模块软件开发和单元测试
  \end{itemize}
\end{itemize}

% ============ 教育背景 ============
\section*{教育背景}

\begin{itemize}
  \item \textbf{德国卡尔斯鲁厄理工学院(KIT)} \hfill
    \textit{机电一体化及信息技术硕士} \hfill
    \textit{2017.04 -- 2020.10}
  \begin{itemize}
    \item 深化方向:工业自动化、机器人技术
    \item 优秀毕业设计:基于\tech{Python}和深度学习的动态眼球追踪系统数据质量优化
  \end{itemize}

  \item \textbf{河北工业大学(211)} \hfill
  
    \textit{机械设计制造及其自动化学士} \hfill
    \textit{2012年9月 -- 2016年7月}
  \begin{itemize}
    \item 优秀毕业设计:人体工程学自动调节座椅
  \end{itemize}
\end{itemize}

% ============ 证书与其他 ============
\section*{证书与其他}

\begin{itemize}
  \item \textbf{外语能力}:英语(雅思:6.5),德语(TestDaf:16/C1)
  \item \textbf{爱好特长}:中长跑、硬笔书法、无动力帆船
  \item \textbf{科世达2022年度优秀员工}
  \item \textbf{科世达AE电子开发部颁奖大会(2022)及年会(2023)主持人}
  \item \textbf{上海防疫期间驻守公司保障项目交付}
\end{itemize}

\end{document}
