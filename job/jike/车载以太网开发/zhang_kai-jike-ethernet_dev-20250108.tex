% !TEX program = xelatex
\documentclass[11pt,a4paper]{ctexart}

% ============ 基础包 ============
\usepackage[margin=2cm]{geometry}
\usepackage{ctex}
\usepackage{fancyhdr}
\usepackage{titlesec}
\usepackage{enumitem}
\usepackage{graphicx}
\usepackage{hyperref}
\usepackage{xcolor}
\usepackage{tabularx}
\usepackage{array}
\usepackage{multirow}
\usepackage{pifont}
\usepackage{wrapfig}
\usepackage{float}

% ============ 页面设置 ============
\geometry{left=1.8cm, right=1.8cm, top=2cm, bottom=2cm}
\pagestyle{fancy}
\fancyhf{}
\fancyfoot[C]{\thepage}
\renewcommand{\headrulewidth}{0pt}

% ============ 颜色定义 ============
\definecolor{primary}{RGB}{0,82,147}      % 汽车蓝
\definecolor{secondary}{RGB}{102,102,102} % 灰色
\definecolor{accent}{RGB}{220,20,60}      % 强调红
\definecolor{techblue}{RGB}{0,114,188}    % 技术蓝
\definecolor{lightgray}{RGB}{240,240,240}

% ============ 字体设置(系统字体) ============
\setCJKmainfont{SimHei}
\setCJKsansfont{SimHei}
\setCJKmonofont{FangSong}

% ============ 标题格式 ============
\titleformat{\section}
  {\Large\bfseries\sffamily\color{primary}}
  {}
  {0em}
  {}
  [{\titlerule[0.5pt]\color{primary!30!black}}]

\titleformat{\subsection}
  {\large\bfseries\sffamily\color{black}}
  {}
  {0em}
  {}

\titlespacing{\section}{0pt}{12pt}{8pt}
\titlespacing{\subsection}{0pt}{8pt}{6pt}

% ============ 列表设置 ============
\setlist{
  leftmargin=0pt,
  itemsep=3pt,
  parsep=0pt,
  topsep=3pt,
  labelsep=8pt
}

% ============ 自定义命令 ============
\newcommand{\highlight}[1]{\textcolor{primary}{\textbf{#1}}}
\newcommand{\tech}[1]{\textcolor{techblue}{\texttt{#1}}}
\newcommand{\skillitem}[2]{\item[\ding{108}] \textbf{#1}: #2}

% ============ 超链接设置 ============
\hypersetup{
  colorlinks=true,
  urlcolor=techblue,
  linkcolor=primary,
  citecolor=secondary
}

\begin{document}

% ============ 个人信息 ============
\begin{minipage}{0.65\textwidth}
  {\Huge\bfseries\sffamily 张凯} \\[8pt]
  {\large\texttt{电话:+86 15122986177} | \texttt{邮箱:\href{mailto:15122986177@163.com}{15122986177@163.com}}} \\[6pt]
  {\large\texttt{上海 | 婚姻状况:已婚 | 出生日期:1992年3月11日}}
\end{minipage}
\hfill
\begin{minipage}{0.3\textwidth}
  \centering
  \includegraphics[width=0.9\textwidth,height=3.5cm,keepaspectratio]{profile_photo.jpg}
\end{minipage}

\vspace{10pt}
{\color{primary}\hrule}
\vspace{12pt}

% ============ 职业概述 ============
\section*{职业概述}

\normalsize
5年汽车嵌入式软件开发经验,熟悉车载以太网通信协议栈,具备\tech{SOME/IP}服务设计、配置和测试验证经验,负责\tech{EthTSyn}模块的\tech{IEEE 802.1AS(gPTP)}时间同步配置,了解\tech{IEEE 802.1Qbv}、\tech{IEEE 802.1Qav}等\tech{TSN}协议及\tech{QoS}保障机制,具备\tech{AUTOSAR}架构集成、车载网络测试验证经验。

% ============ 关键能力 ============
\section*{关键能力}

\begin{itemize}
  \skillitem{车载以太网通信}{精通\tech{SOME/IP}服务完整配置流程(\tech{SoAd, SD, PduR}),熟悉\tech{DoIP}、\tech{XCP on Ethernet}、\tech{UDP NM}等协议,了解\tech{TCP/IP}、\tech{VLAN}、\tech{IPv4/v6}等底层协议栈}
  \skillitem{TSN协议栈}{负责\tech{EthTSyn}模块的\tech{IEEE 802.1AS(gPTP)}时间同步配置,了解\tech{IEEE 802.1Qbv}时间感知整形、\tech{IEEE 802.1Qav}信用整形等\tech{QoS}保障机制}
  \skillitem{以太网测试与验证}{熟练使用\tech{CANoe}配合\tech{VN5650}进行\tech{SOME/IP}服务通信测试和\tech{TC8}标准测试,熟悉\tech{Upper Tester}测试规范}
  \skillitem{AUTOSAR架构基础}{熟悉\tech{Classic AUTOSAR}架构,具备\tech{BSW}、\tech{RTE}、\tech{SWC}集成经验,掌握\tech{Davinci Configurator}中的OS时序挂载和任务调度机制}
  \skillitem{开发与调试能力}{精通\tech{Python}脚本开发和\tech{C}语言编程,熟练使用\tech{Lauterbach}、\tech{iSYSTEM winIDEA}进行\tech{C}代码级调试}
\end{itemize}

\subsection*{技术技能矩阵}

\begin{tabularx}{\textwidth}{@{}p{3.2cm}X@{}}
  \textbf{以太网协议} & \tech{SOME/IP}, \tech{DoIP}, \tech{XCP on Ethernet}, \tech{TCP/IP}, \tech{UDP NM}, \tech{VLAN}, \tech{IPv4/v6} \\
  \textbf{TSN协议} & \tech{IEEE 802.1AS(gPTP)}, \tech{IEEE 802.1Qbv}, \tech{IEEE 802.1Qav} \\
  \textbf{测试工具} & \tech{CANoe+VN5650}, \tech{TC8 Upper Tester}, \tech{vTESTstudio} \\
  \textbf{开发平台} & \tech{AUTOSAR CP}, \tech{C}, \tech{Python} \\
  \textbf{调试工具} & \tech{Lauterbach Trace32}, \tech{iSYSTEM winIDEA}
\end{tabularx}

% ============ 工作经历 ============
\section*{工作经历}

\subsection*{沃尔沃汽车(亚太)投资控股有限公司 Volvo Car (Asia Pacific) Investment Holding Co., Ltd.}
\textbf{中央计算平台集成主管工程师} \hfill \textit{2023年11月 -- 至今,中国上海}

\subsubsection*{SPA3/GPA高性能计算平台集成|车载以太网中间件配置负责人}
\hfill \textit{2023年11月 -- 2025年7月}

\begin{itemize}
  \item[\ding{51}] \highlight{SOME/IP服务配置与验证}:主导车载以太网\tech{SOME/IP}服务的完整配置流程,在\tech{Davinci Configurator}中完成\tech{SoAd}(Socket Adapter)、\tech{SD}(Service Discovery)、\tech{PduR}等模块配置,负责服务发布/订阅关系的配置管理
  \item[\ding{51}] \highlight{TSN时间同步配置}:负责\tech{EthTSyn}模块的\tech{IEEE 802.1AS(gPTP)}时间同步配置,设置主时钟同步、时间戳同步和时钟精度参数,确保车载以太网网络的确定性通信保障
  \item[\ding{51}] \highlight{以太网通信测试验证}:使用\tech{CANoe}配合\tech{VN5650}进行\tech{SOME/IP}服务通信测试,验证服务发现(\tech{SD})机制、\tech{Offer}/\tech{Find}服务流程、\tech{Subscribe}/\tech{Subscribe Ack}订阅确认流程,以及实际数据传输的时序要求
\end{itemize}

\subsubsection*{CI/CD架构与工具链开发|流水线设计开发者}
\hfill \textit{2025年8月 -- 至今}

\begin{itemize}
  \item[\ding{51}] \highlight{CI/CD流水线架构与脚本开发}:设计企业级\tech{CI/CD}流水线系统,基于\tech{Jenkins-Gerrit}技术栈实现代码质量门禁系统,开发\tech{Groovy}脚本并集成\tech{ktlint, detekt}等静态检查工具,确保代码质量合规
  \item[\ding{51}] \highlight{并发架构设计}:设计主从\tech{Job}并发执行架构,使用多进程并发,提升\tech{CI/CD}流水线执行效率\textbf{30\%}
\end{itemize}

\subsection*{科世达(上海)机电有限公司 KOSTAL (Shanghai) Mechatronic Co., Ltd.}
\textbf{车身域控制器开发工程师} \hfill \textit{2020年12月 -- 2023年11月,中国上海}

\subsubsection*{车身域控制器集成负责人}
\hfill \textit{2020年12月 -- 2023年11月}

\begin{itemize}
  \item[\ding{51}] \highlight{车载网络集成}:负责车身域控制器的\tech{CAN/LIN/以太网}通信集成,主导\tech{SWC}间\tech{RTE}接口设计和连接,确保车载网络通信的可靠性和实时性
  \item[\ding{51}] \highlight{自动化测试平台开发}:基于\tech{Python}开发自动化测试脚本生成工具,根据客户通信矩阵生成\tech{CANoe CAPL}脚本和\tech{vTESTstudio}配置,构建完整的自动化测试工程,实现从测试用例设计到执行的全流程自动化
  \item[\ding{51}] \highlight{通信协议测试验证}:针对\tech{CAN/LIN/以太网}通信协议设计测试用例,使用\tech{CANoe}进行车载网络通信测试验证,验证通信的可靠性和实时性
  \item[\ding{51}] \highlight{软件集成与发布}:进行完整的编译、烧录、测试流程,负责软件集成测试和系统级问题定位,建立软件版本与需求关系管理,确保软件符合功能需求
  \item[\ding{51}] \highlight{质量管控与问题跟踪}:作为技术接口人协调多个团队,推动产品问题解决,建立质量问题分派和跟踪机制,确保项目按时交付
\end{itemize}

\subsubsection*{AUTOSAR车身控制器项目经验}
\hfill \textit{2021年1月 -- 2023年8月}

\begin{itemize}
  \item[\ding{108}] \textbf{长城汽车} - 欧拉好猫/黑猫系列\tech{KBCM}项目|车身域控制器集成负责人
  \begin{itemize}
    \item 主导\tech{AUTOSAR CP}架构下的车身控制器集成开发,负责\tech{BLE, TBOX, VCU, ESCL}等安全关键模块的通信集成,符合\tech{ISO 26262}功能安全要求
    \item 负责\tech{NVM Manager}模块开发,实现配置数据和诊断数据的持久化存储
    \item 进行\tech{OS}时序挂载和任务调度优化,提升系统响应速度和稳定性
    \item 负责\tech{IOHWab}模块开发,实现硬件抽象层与上层应用的接口, 使SWC开发与硬件平台解耦
  \end{itemize}

  \item[\ding{108}] \textbf{理想汽车} - 车和家系列\tech{KBCM}项目|集成开发负责人
  \begin{itemize}
    \item 负责新能源汽车车身控制器的\tech{AUTOSAR}架构集成和功能安全验证
  \end{itemize}

  \item[\ding{108}] \textbf{江西五十铃} - \tech{PEPS}无钥匙进入系统|集成开发负责人
  \begin{itemize}
    \item 主导车辆防盗安全系统的通信协议集成和测试验证,确保\tech{PEPS}系统满足\tech{ISO 26262 ASIL-B}功能安全等级要求
  \end{itemize}

  \item[\ding{108}] \textbf{宇通客车} - \tech{KBCM}车身控制器项目|软件开发工程师
  \begin{itemize}
    \item 负责商用车车身控制器的内灯模块\tech{SWC}开发、单元测试和系统集成
  \end{itemize}
\end{itemize}

% ============ 教育背景 ============
\section*{教育背景}

\begin{itemize}
  \item \textbf{德国卡尔斯鲁厄理工学院(KIT)} \hfill
    \textit{机电一体化及信息技术硕士} \hfill
    \textit{2017.04 -- 2020.10}
  \begin{itemize}
    \item 深化方向:工业自动化、机器人技术
    \item 优秀毕业设计:基于\tech{Python}和深度学习的动态眼球追踪系统数据质量优化
  \end{itemize}

  \item \textbf{河北工业大学(211)} \hfill
    \textit{机械设计制造及其自动化学士} \hfill
    \textit{2012年9月 -- 2016年7月}
  \begin{itemize}
    \item 优秀毕业设计:人体工程学自动调节座椅
  \end{itemize}
\end{itemize}

% ============ 证书与其他 ============
\section*{证书与其他}

\begin{itemize}
  \item \textbf{外语能力}:英语(雅思:6.5),德语(TestDaf:16/C1)
  \item \textbf{爱好特长}:中长跑、硬笔书法、无动力帆船
  \item \textbf{科世达2022年度优秀员工}
  \item \textbf{科世达AE电子开发部颁奖大会(2022)及年会(2023)主持人}
  \item \textbf{上海防疫期间驻守公司保障项目交付}
\end{itemize}

\end{document}
