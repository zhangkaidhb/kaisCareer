% !TEX program = xelatex
\documentclass[11pt,a4paper]{ctexart}

% ============ 基础包 ============
\usepackage[margin=2cm]{geometry}
\usepackage{ctex}
\usepackage{fancyhdr}
\usepackage{titlesec}
\usepackage{enumitem}
\usepackage{graphicx}
\usepackage{hyperref}
\usepackage{xcolor}
\usepackage{tabularx}
\usepackage{array}
\usepackage{multirow}
\usepackage{pifont}
\usepackage{wrapfig}
\usepackage{float}

% ============ 页面设置 ============
\geometry{left=1.8cm, right=1.8cm, top=2cm, bottom=2cm}
\pagestyle{fancy}
\fancyhf{}
\fancyfoot[C]{\thepage}
\renewcommand{\headrulewidth}{0pt}

% ============ 颜色定义 ============
\definecolor{primary}{RGB}{0,82,147}      % 汽车蓝
\definecolor{secondary}{RGB}{102,102,102} % 灰色
\definecolor{accent}{RGB}{220,20,60}      % 强调红
\definecolor{techblue}{RGB}{0,114,188}    % 技术蓝
\definecolor{lightgray}{RGB}{240,240,240}

% ============ 字体设置(系统字体) ============
\setCJKmainfont{SimHei}
\setCJKsansfont{SimHei}
\setCJKmonofont{FangSong}

% ============ 标题格式 ============
\titleformat{\section}
  {\Large\bfseries\sffamily\color{primary}}
  {}
  {0em}
  {}
  [{\titlerule[0.5pt]\color{primary!30!black}}]

\titleformat{\subsection}
  {\large\bfseries\sffamily\color{black}}
  {}
  {0em}
  {}

\titlespacing{\section}{0pt}{12pt}{8pt}
\titlespacing{\subsection}{0pt}{8pt}{6pt}

% ============ 列表设置 ============
\setlist{
  leftmargin=0pt,
  itemsep=3pt,
  parsep=0pt,
  topsep=3pt,
  labelsep=8pt
}

% ============ 自定义命令 ============
\newcommand{\highlight}[1]{\textcolor{primary}{\textbf{#1}}}
\newcommand{\tech}[1]{\textcolor{techblue}{\texttt{#1}}}
\newcommand{\skillitem}[2]{\item[\ding{108}] \textbf{#1}: #2}

% ============ 超链接设置 ============
\hypersetup{
  colorlinks=true,
  urlcolor=techblue,
  linkcolor=primary,
  citecolor=secondary
}

\begin{document}

% ============ 个人信息 ============
\begin{minipage}{0.65\textwidth}
  {\Huge\bfseries\sffamily 张凯} \\[8pt]
  {\large\texttt{电话:+86 15122986177} | \texttt{邮箱:\href{mailto:15122986177@163.com}{15122986177@163.com}}} \\[6pt]
  {\large\texttt{上海 | 婚姻状况:已婚 | 出生日期:1992年3月11日}}
\end{minipage}
\hfill
\begin{minipage}{0.3\textwidth}
  \centering
  \includegraphics[width=0.9\textwidth,height=3.5cm,keepaspectratio]{profile_photo.jpg}
\end{minipage}

\vspace{10pt}
{\color{primary}\hrule}
\vspace{12pt}

% ============ 职业概述 ============
\section*{职业概述}

\normalsize
5年以上\tech{AUTOSAR}软件开发经验,擅长智能座舱\tech{MCU}软件架构设计与应用层开发,精通\tech{SWC}(\tech{Runnable})、\tech{CDD}复杂驱动开发及\tech{BSW}基础软件配置集成,具备高性能中央计算平台(\tech{HI}基于Infineon TC397)和车身域控制器集成经验,熟练使用\tech{Vector DaVinci}工具链,熟悉\tech{CAN}、\tech{UDS}、\tech{NM}、\tech{DoIP}等座舱通信协议。

% ============ 关键能力 ============
\section*{关键能力}

\begin{itemize}
  \skillitem{智能座舱MCU架构}{精通智能座舱\tech{MCU}软件架构设计,熟悉\tech{AUTOSAR SWC}应用层开发(\tech{Runnable}实体、\tech{Data Type}、\tech{Interface}定义),具备复杂驱动\tech{CDD}开发经验,能够进行\tech{RTE}接口配置和代码生成}
  \skillitem{AUTOSAR基础软件配置}{熟练掌握\tech{AUTOSAR BSW}标准模块配置,包括通信(\tech{CAN, LIN, Eth})、诊断(\tech{UDS on CAN/DoIP})、存储(\tech{NVM})、模式管理(\tech{Mode Management})、网络管理(\tech{NM})等核心功能模块}
  \skillitem{Vector DaVinci工具链}{精通\tech{DaVinci Developer}进行\tech{SWC}应用层开发和\tech{RTE}配置,熟练使用\tech{DaVinci Configurator}进行\tech{BSW}模块配置和代码生成,掌握\tech{ECUExtract}合规检查和\tech{ARXML}文件管理}
  \skillitem{座舱通信协议}{精通\tech{CAN}总线通信协议,熟悉\tech{UDS}诊断协议(\tech{ISO 14229}、\tech{ISO 15765})、\tech{NM}网络管理(\tech{OSEK/AUTOSAR NM})、\tech{DoIP}(\tech{UDS on Ethernet})等座舱常用协议栈}
  \skillitem{工具链开发与自动化}{精通\tech{Python}脚本开发,主导\tech{yaml2arxml}转换工具和\tech{SWC}配置管理工具开发,具备自动化测试平台开发经验}
  \skillitem{系统集成与调试}{熟练使用\tech{Lauterbach Trace32}、\tech{iSYSTEM winIDEA}进行\tech{C}代码级调试,具备软件集成测试和系统级问题定位能力}
\end{itemize}

\subsection*{技术技能矩阵}

\begin{tabularx}{\textwidth}{@{}p{3.2cm}X@{}}
  \textbf{AUTOSAR开发} & \tech{SWC应用层开发}, \tech{CDD复杂驱动}, \tech{BSW配置集成}, \tech{RTE接口连接}, \tech{OS时序挂载} \\
  \textbf{Vector工具链} & \tech{DaVinci Developer}, \tech{DaVinci Configurator}, \tech{CANoe}, \tech{vTESTstudio} \\
  \textbf{座舱协议} & \tech{CAN}, \tech{UDS}, \tech{NM(OSEK/AUTOSAR)}, \tech{DoIP}, \tech{LIN} \\
  \textbf{编程语言} & \tech{C}, \tech{Python}, \tech{MATLAB Stateflow} \\
  \textbf{调试工具} & \tech{Lauterbach Trace32}, \tech{iSYSTEM winIDEA}, \tech{CANoe}
\end{tabularx}

% ============ 工作经历 ============
\section*{工作经历}

\subsection*{沃尔沃汽车(亚太)投资控股有限公司 Volvo Car (Asia Pacific) Investment Holding Co., Ltd.}
\textbf{中央计算平台集成主管工程师} \hfill \textit{2023年11月 -- 至今,中国上海}

\subsubsection*{SPA3/GPA高性能计算平台集成|智能座舱MCU集成负责人}
\hfill \textit{2023年11月 -- 2025年7月}

\begin{itemize}
  \item[\ding{51}] \highlight{智能座舱MCU软件架构集成}:负责高性能计算平台\tech{HI}(基于Infineon TC397)的智能座舱\tech{MCU}软件架构集成,主导\tech{AUTOSAR CP}架构下的\tech{SWC}应用层开发和\tech{BSW}基础软件配置,涵盖座舱域控制、车身控制、网络管理等核心功能
  \item[\ding{51}] \highlight{SWC应用层开发与配置}:使用\tech{DaVinci Developer}进行\tech{SWC}应用层开发,负责\tech{Runnable}实体设计、\tech{Data Type}定义、\tech{Sender-Receiver}和\tech{Client-Server}接口配置,主导\tech{RTE}接口连接和代码生成,实现座舱功能的模块化设计
  \item[\ding{51}] \highlight{CDD复杂驱动开发}:负责座舱特定硬件的\tech{CDD}(\tech{Complex Device Driver})开发,包括\tech{IoHwAb}(\tech{IO Hardware Abstraction})模块、\tech{Auth}安全认证模块、\tech{NvmManager}存储管理模块等基于\tech{C}语言的底层驱动开发
  \item[\ding{51}] \highlight{BSW基础软件配置}:使用\tech{DaVinci Configurator}进行\tech{BSW}标准模块配置,包括\tech{CAN}通信(\tech{CanIf, CanTp, PduR})、\tech{UDS}诊断(\tech{DEM, Dcm, UDSTp})、\tech{NVM}存储(\tech{NvM, MemIf, Fee})、\tech{NM}网络管理、\tech{ComM}通信管理等核心功能模块
  \item[\ding{51}] \highlight{工具链开发与自动化}:主导基于\tech{Python}的\tech{yaml2arxml}转换工具和\tech{SWC}配置管理工具开发,实现从\tech{yaml}脚本配置文件到\tech{ARXML}和代码的全流程分布式管理自动化,替代\tech{Davinci Developer}核心功能,显著提升配置效率
  \item[\ding{51}] \highlight{DoIP诊断配置}:负责\tech{DoIP}(\tech{UDS on Ethernet})诊断协议配置,在\tech{Davinci Configurator}中完成\tech{SoAd}、\tech{DoIP}模块配置,支持智能座舱的以太网诊断功能
  \item[\ding{51}] \highlight{系统调试与验证}:使用\tech{CANoe}进行座舱\tech{CAN/LIN}通信测试,配合\tech{Lauterbach}调试器进行\tech{C}代码级调试,验证\tech{SWC}功能正确性和\tech{BSW}模块配置正确性,保障智能座舱\tech{MCU}软件的系统稳定性
\end{itemize}

\subsubsection*{CI/CD架构与工具链开发|流水线设计开发者}
\hfill \textit{2025年8月 -- 至今}

\begin{itemize}
  \item[\ding{51}] \highlight{CI/CD流水线架构与脚本开发}:设计企业级\tech{CI/CD}流水线系统,基于\tech{Jenkins-Gerrit}技术栈实现代码质量门禁系统,开发\tech{Groovy}脚本并集成静态检查工具,确保代码质量合规
  \item[\ding{51}] \highlight{并发架构设计}:设计主从\tech{Job}并发执行架构,使用多进程并发,提升\tech{CI/CD}流水线执行效率\textbf{30\%}
  \item[\ding{51}] \highlight{可配置化开发}:设计\tech{YAML}配置驱动的测试框架,支持同一个\tech{Jenkins Job}的多版本\tech{pipeline}脚本的动态加载和执行
\end{itemize}

\subsection*{科世达(上海)机电有限公司 KOSTAL (Shanghai) Mechatronic Co., Ltd.}
\textbf{车身域控制器开发工程师} \hfill \textit{2020年12月 -- 2023年11月,中国上海}

\subsubsection*{车身域控制器集成负责人}
\hfill \textit{2020年12月 -- 2023年11月}

\begin{itemize}
  \item[\ding{51}] \highlight{AUTOSAR架构集成}:负责车身域控制器的\tech{AUTOSAR CP}架构集成,主导\tech{SWC}间\tech{RTE}接口设计和连接、\tech{OS}时序挂载和基础软件组件集成,确保软件架构和接口连接符合\tech{SWC}功能需求
  \item[\ding{51}] \highlight{SWC模块开发}:负责车身域控制器的部分\tech{SWC}模块开发,包括\tech{Auth}安全认证模块(基于\tech{C}语言)、\tech{NvmManager}模块(基于\tech{C}语言)、\tech{IoHwAb}模块(基于\tech{C}语言)、内灯模块(基于\tech{MATLAB Stateflow})
  \item[\ding{51}] \highlight{软件集成与发布}:进行完整的编译、烧录、测试流程,负责软件集成测试和系统级问题定位,建立软件版本与需求关系管理,确保软件符合功能需求
  \item[\ding{51}] \highlight{质量管控与问题跟踪}:作为技术接口人协调多个团队,推动产品问题解决,建立质量问题分派和跟踪机制,确保项目按时交付
  \item[\ding{51}] \highlight{自动化测试工具开发}:基于\tech{Python}开发自动化测试脚本生成工具,根据客户通信矩阵生成\tech{CANoe CAPL}脚本和\tech{vTESTstudio}配置,构建自动化测试框架提升集成测试效率
\end{itemize}

\subsubsection*{AUTOSAR车身控制器项目经验}
\hfill \textit{2021年1月 -- 2023年8月}

\begin{itemize}
  \item[\ding{108}] \textbf{长城汽车} - 欧拉好猫/黑猫系列\tech{KBCM}项目|车身域控制器集成负责人
  \begin{itemize}
    \item 主导\tech{AUTOSAR CP}架构下的车身控制器集成开发,负责\tech{BLE, TBOX, VCU, ESCL}等安全关键模块的通信集成,符合\tech{ISO 26262}功能安全要求
    \item 建立安全关键系统的通信测试验证体系,确保车身域控制器关键模块在\tech{ISO 26262 ASIL-D}等级要求下的功能安全性
  \end{itemize}

  \item[\ding{108}] \textbf{理想汽车} - 车和家系列\tech{KBCM}项目|集成开发负责人
  \begin{itemize}
    \item 负责新能源汽车车身控制器的\tech{AUTOSAR}架构集成和功能安全验证
  \end{itemize}

  \item[\ding{108}] \textbf{江西五十铃} - \tech{PEPS}无钥匙进入系统|集成开发负责人
  \begin{itemize}
    \item 主导车辆防盗安全系统的通信协议集成和测试验证,确保\tech{PEPS}系统满足\tech{ISO 26262 ASIL-B}功能安全等级要求
  \end{itemize}

  \item[\ding{108}] \textbf{宇通客车} - \tech{KBCM}车身控制器项目|软件开发工程师
  \begin{itemize}
    \item 负责商用车车身控制器的内灯模块\tech{SWC}开发、单元测试和系统集成
  \end{itemize}
\end{itemize}

% ============ 教育背景 ============
\section*{教育背景}

\begin{itemize}
  \item \textbf{德国卡尔斯鲁厄理工学院(KIT)} \hfill
    \textit{机电一体化及信息技术硕士} \hfill
    \textit{2017.04 -- 2020.10}
  \begin{itemize}
    \item 深化方向:工业自动化、机器人技术
    \item 优秀毕业设计:基于\tech{Python}和深度学习的动态眼球追踪系统数据质量优化
  \end{itemize}

  \item \textbf{河北工业大学(211)} \hfill
    \textit{机械设计制造及其自动化学士} \hfill
    \textit{2012年9月 -- 2016年7月}
  \begin{itemize}
    \item 优秀毕业设计:人体工程学自动调节座椅
  \end{itemize}
\end{itemize}

% ============ 证书与其他 ============
\section*{证书与其他}

\begin{itemize}
  \item \textbf{外语能力}:英语(雅思:6.5),德语(TestDaf:16/C1)
  \item \textbf{爱好特长}:中长跑、硬笔书法、无动力帆船
  \item \textbf{科世达2022年度优秀员工}
  \item \textbf{科世达AE电子开发部颁奖大会(2022)及年会(2023)主持人}
  \item \textbf{上海防疫期间驻守公司保障项目交付}
\end{itemize}

\end{document}
