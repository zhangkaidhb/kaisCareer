% !TEX program = xelatex
\documentclass[11pt,a4paper]{ctexart}

% ============ 基础包 ============
\usepackage[margin=2cm]{geometry}
\usepackage{ctex}
\usepackage{fancyhdr}
\usepackage{titlesec}
\usepackage{enumitem}
\usepackage{graphicx}
\usepackage{hyperref}
\usepackage{xcolor}
\usepackage{tabularx}
\usepackage{array}
\usepackage{multirow}
\usepackage{pifont}

% ============ 页面设置 ============
\geometry{left=1.8cm, right=1.8cm, top=2cm, bottom=2cm}
\pagestyle{fancy}
\fancyhf{}
\fancyfoot[C]{\thepage}
\renewcommand{\headrulewidth}{0pt}

% ============ 颜色定义 ============
\definecolor{primary}{RGB}{0,82,147}      % 汽车蓝
\definecolor{secondary}{RGB}{102,102,102} % 灰色
\definecolor{accent}{RGB}{220,20,60}      % 强调红
\definecolor{techblue}{RGB}{0,114,188}    % 技术蓝
\definecolor{lightgray}{RGB}{240,240,240}

% ============ 字体设置(系统字体) ============
\setCJKmainfont{SimHei}
\setCJKsansfont{SimHei}
\setCJKmonofont{FangSong}

% ============ 标题格式 ============
\titleformat{\section}
  {\Large\bfseries\sffamily\color{primary}}
  {}
  {0em}
  {}
  [{\titlerule[0.5pt]\color{primary!30!black}}]

\titleformat{\subsection}
  {\large\bfseries\sffamily\color{black}}
  {}
  {0em}
  {}

\titlespacing{\section}{0pt}{12pt}{8pt}
\titlespacing{\subsection}{0pt}{8pt}{6pt}

% ============ 列表设置 ============
\setlist{
  leftmargin=0pt,
  itemsep=3pt,
  parsep=0pt,
  topsep=3pt,
  labelsep=8pt
}

% ============ 自定义命令 ============
\newcommand{\highlight}[1]{\textcolor{primary}{\textbf{#1}}}
\newcommand{\tech}[1]{\textcolor{techblue}{\texttt{#1}}}
\newcommand{\skillitem}[2]{\item[\ding{108}] \textbf{#1}: #2}

% ============ 超链接设置 ============
\hypersetup{
  colorlinks=true,
  urlcolor=techblue,
  linkcolor=primary,
  citecolor=secondary
}

\begin{document}

% ============ 个人信息 ============
\begin{center}
  {\Huge\bfseries\sffamily 张凯} \\[8pt]
  {\large\texttt{电话:+86 15122986177} | \texttt{邮箱:\href{mailto:15122986177@163.com}{15122986177@163.com}}} \\[6pt]
  {\large\texttt{上海 | 婚姻状况:已婚 | 出生日期:1992年3月11日}} \\[4pt]
  {\normalsize\color{secondary}汽车电子软件架构师 | AUTOSAR与确定性系统专家}
\end{center}

\vspace{10pt}
{\color{primary}\hrule}
\vspace{12pt}

% ============ 职业概述 ============
\section*{职业概述}

\normalsize
5年汽车电子软件开发经验,专注于跨域融合软件架构设计、确定性中间件开发、实时通信系统。精通AUTOSAR AP/CP架构、TSN网络调度、车载以太网通信中间件,具备从架构设计到实际部署的全栈开发能力。

% ============ 关键能力 ============
\section*{关键能力(汽车软件方向)}

\begin{itemize}
  \skillitem{AUTOSAR架构与系统集成}{精通AUTOSAR CP架构,熟悉Adaptive AUTOSAR,具备HPC平台和车身域控制器集成经验,精通车载网络通信(CAN/LIN/以太网)}
  \skillitem{确定性通信中间件}{主导SOME/IP服务配置,深入理解TSN协议族(IEEE802.1AS gPTP, 802.1Qbv, 802.1Qav)及QoS保障机制,具备车载以太网通信经验}
  \skillitem{软件开发能力}{精通Python、MATLAB Stateflow、C编程,具备汽车电子SWC开发经验、AUTOSAR架构调试能力和UML架构设计能力}
  \skillitem{工具链与自动化}{精通Vector工具链,具备CI/CD流水线设计和自动化脚本开发经验,熟悉Linux环境}
\end{itemize}

\subsection*{技术技能矩阵}

\begin{tabularx}{\textwidth}{@{}p{3.2cm}X@{}}
  \textbf{编程语言} & Python, Groovy, C, MATLAB(Stateflow) \\
  \textbf{Vector工具链} & DaVinci Developer\&Configurator, CANoe, vTESTstudio \\
  \textbf{CI/CD工具} & Jenkins, Gerrit, Gradle, MongoDB, Groovy脚本, Git \\
  \textbf{调试工具} & iSYSTEM winIDEA, Lauterbach Trace32
\end{tabularx}

% ============ 工作经历 ============
\section*{工作经历}

\subsection*{沃尔沃汽车(亚太)投资控股有限公司 Volvo Car (Asia Pacific) Investment Holding Co., Ltd.}
\textbf{汽车电子软件架构师} \hfill \textit{2023年11月 -- 至今,中国上海}

\subsubsection*{SPA3/GPA高性能计算平台集成|AUTOSAR工具链开发负责人}
\hfill \textit{2023年11月 -- 2025年7月}

\begin{itemize}
  \item[\ding{51}] \highlight{AUTOSAR工具链开发与集成}:负责高性能计算平台的AUTOSAR CP架构集成,主导yaml2arxml转换工具和SWC配置管理工具开发,实现从yaml脚本配置文件到ARXML和代码的全流程分布式管理自动化,替代Davinci Developer核心功能,显著提升综合配置效率
  \item[\ding{51}] \highlight{ARXML文件处理与合规检查}:负责ECUExtract的合规检查和错误修复,确保AUTOSAR架构一致性和配置文件的完整性
  \item[\ding{51}] \highlight{车载以太网中间件配置}:主导\tech{SOME/IP}服务在Davinci Configurator中的完整配置流程,包含\tech{SoAd, SD, BswM, EcuC, ComM, PduR}等模块配置,负责\tech{EthTSyn}模块的\tech{IEEE802.1AS(gPTP)}时间同步配置,确保服务正确部署和确定性通信保障
  \item[\ding{51}] \highlight{通信验证与测试}:使用\tech{CANoe}配合\tech{VN5650}进行SOME/IP服务通信测试,验证车载以太网通信机制和时序要求
  \item[\ding{51}] \highlight{系统调试与部署}:通过Linux环境配置和劳德巴赫调试器进行C代码级调试,保障系统集成稳定性
\end{itemize}

\subsubsection*{CI/CD架构与工具链开发|流水线设计负责人}
\hfill \textit{2025年8月 -- 至今}

\begin{itemize}
  \item[\ding{51}] \highlight{CI/CD流水线架构与脚本开发}:支持设计企业级CI/CD流水线系统,基于\tech{Gerrit-Jenkins-Gradle}技术栈实现代码质量门禁系统,使用Groovy语言开发Jenkins Job脚本并集成\tech{ktlint, detekt}等静态检查工具和Gradle编译器,确保代码质量合规
  \item[\ding{51}] \highlight{并发架构设计}:设计主从Job并发执行架构,使用多进程并发,提升CI/CD流水线执行效率\textbf{30\%}
  \item[\ding{51}] \highlight{数据管理方案}:基于数据库服务设计检查结果存储和仲裁机制,实现基于RPC实现的跨服务器代码检查结果分批记录与集中仲裁,提升门禁系统的灵活性
  \item[\ding{51}] \highlight{可配置化开发}:设计YAML配置驱动的测试框架,支持同一个Jenkins Job的多版本pipeline脚本的动态加载和执行
\end{itemize}

\subsection*{科世达(上海)机电有限公司 KOSTAL (Shanghai) Mechatronic Co., Ltd.}
\textbf{车身域控制器开发工程师} \hfill \textit{2020年12月 -- 2023年11月,中国上海}

\subsubsection*{科世达车身域控制器CEM项目|AUTOSAR集成与中间件开发负责人}
\hfill \textit{2022年12月 -- 2023年11月}

\begin{itemize}
  \item[\ding{51}] \highlight{AUTOSAR架构集成}:负责车身域控制器的\tech{AUTOSAR CP}架构集成,主导SWC间通信配置和中间件组件集成
  \item[\ding{51}] \highlight{系统级调试与部署}:建立完整的编译-烧录-测试流程,负责软件集成测试和系统级问题定位,确保系统稳定性
  \item[\ding{51}] \highlight{测试工具链开发}:设计产线终检程序和EMC测试上位机软件,构建自动化测试框架提升测试效率
  \item[\ding{51}] \highlight{跨团队技术协调}:作为技术接口人推动软件问题解决,确保项目按时交付和后期维护支持
\end{itemize}

\subsubsection*{车载网络自动化测试平台开发|Vector工具链专家}
\hfill \textit{2023年1月 -- 2023年6月}

\begin{itemize}
  \item[\ding{51}] \highlight{自动化工具链开发}:基于\tech{Python}开发自动化测试脚本生成工具,根据客户通信矩阵生成\tech{CANoe CAPL}脚本和\tech{vTESTstudio}配置
  \item[\ding{51}] \highlight{通信协议测试验证}:针对CAN/LIN/以太网通信协议设计测试用例,验证车载网络通信的可靠性和实时性
  \item[\ding{51}] \highlight{测试工程构建}:构建完整的vTESTstudio自动化测试工程,实现从测试用例设计到执行的全流程自动化
  \item[\ding{51}] \highlight{测试技术推广}:在团队内推广自动化测试方法论,提升团队测试效率和质量保证能力
\end{itemize}

\subsubsection*{AUTOSAR车身控制器项目经验}
\hfill \textit{2021年1月 -- 2023年8月}

\begin{itemize}
  \item[\ding{108}] \textbf{长城汽车} - 欧拉好猫/黑猫系列KBCM项目|车身域控制器集成负责人
  \begin{itemize}
    \item 主导AUTOSAR CP架构下的车身控制器集成开发,负责\tech{BLE, TBOX, VCU, ESCL}等安全关键模块的通信集成,符合\tech{ISO 26262}功能安全要求
    \item 建立安全关键系统的通信测试验证体系,确保车身域控制器在ASIL等级要求下的功能安全性
  \end{itemize}

  \item[\ding{108}] \textbf{理想汽车} - 车和家系列KBCM项目|集成开发负责人
  \begin{itemize}
    \item 负责新能源汽车车身控制器的AUTOSAR架构集成和功能安全验证
  \end{itemize}

  \item[\ding{108}] \textbf{江西五十铃} - PEPS无钥匙进入系统|集成开发负责人
  \begin{itemize}
    \item 主导车辆防盗安全系统的通信协议集成和测试验证,确保PEPS系统满足\tech{ISO 26262 ASIL-B}功能安全等级要求
  \end{itemize}

  \item[\ding{108}] \textbf{宇通客车} - KBCM车身控制器项目|软件开发工程师
  \begin{itemize}
    \item 负责商用车车身控制器的内灯模块开发和系统集成
  \end{itemize}
\end{itemize}

% ============ 教育背景 ============
\section*{教育背景}

\begin{itemize}
  \item \textbf{德国卡尔斯鲁厄理工学院(KIT)} \hfill
    \textit{机电一体化及信息技术硕士} \hfill
    \textit{2017.04 -- 2020.10}
  \begin{itemize}
    \item 深化方向:工业自动化、机器人技术
    \item 优秀毕业设计:基于Python和深度学习的动态眼球追踪系统数据质量优化
  \end{itemize}

  \item \textbf{河北工业大学(211)} \hfill
    \textit{机械设计制造及其自动化学士} \hfill
    \textit{2012年9月 -- 2016年7月}
  \begin{itemize}
    \item 优秀毕业设计:人体工程学自动调节座椅
  \end{itemize}
\end{itemize}

% ============ 证书与其他 ============
\section*{证书与其他}

\begin{itemize}
  \item \textbf{外语能力}:英语(雅思:6.5),德语(TestDaf:16/C1)
  \item \textbf{爱好特长}:中长跑、硬笔书法、无动力帆船
  \item \textbf{科世达2022年度优秀员工}
  \item \textbf{科世达AE电子开发部年会主持人(2次)}
  \item \textbf{上海防疫期间驻守公司保障项目交付}
\end{itemize}

\end{document}