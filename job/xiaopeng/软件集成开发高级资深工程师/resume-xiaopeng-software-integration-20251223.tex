% !TEX program = xelatex
\documentclass[11pt,a4paper]{ctexart}

% ============ 基础包 ============
\usepackage[margin=2cm]{geometry}
\usepackage{ctex}
\usepackage{fancyhdr}
\usepackage{titlesec}
\usepackage{enumitem}
\usepackage{graphicx}
\usepackage{hyperref}
\usepackage{xcolor}
\usepackage{tabularx}
\usepackage{array}
\usepackage{multirow}
\usepackage{pifont}
\usepackage{wrapfig}
\usepackage{float}

% ============ 页面设置 ============
\geometry{left=1.8cm, right=1.8cm, top=2cm, bottom=2cm}
\pagestyle{fancy}
\fancyhf{}
\fancyfoot[C]{\thepage}
\renewcommand{\headrulewidth}{0pt}

% ============ 颜色定义 ============
\definecolor{primary}{RGB}{0,82,147}      % 汽车蓝
\definecolor{secondary}{RGB}{102,102,102} % 灰色
\definecolor{accent}{RGB}{220,20,60}      % 强调红
\definecolor{techblue}{RGB}{0,114,188}    % 技术蓝
\definecolor{lightgray}{RGB}{240,240,240}

% ============ 字体设置(系统字体) ============
\setCJKmainfont{SimHei}
\setCJKsansfont{SimHei}
\setCJKmonofont{FangSong}

% ============ 标题格式 ============
\titleformat{\section}
  {\Large\bfseries\sffamily\color{primary}}
  {}
  {0em}
  {}
  [{\titlerule[0.5pt]\color{primary!30!black}}]

\titleformat{\subsection}
  {\large\bfseries\sffamily\color{black}}
  {}
  {0em}
  {}

\titlespacing{\section}{0pt}{12pt}{8pt}
\titlespacing{\subsection}{0pt}{8pt}{6pt}

% ============ 列表设置 ============
\setlist{
  leftmargin=0pt,
  itemsep=3pt,
  parsep=0pt,
  topsep=3pt,
  labelsep=8pt
}

% ============ 自定义命令 ============
\newcommand{\highlight}[1]{\textcolor{primary}{\textbf{#1}}}
\newcommand{\tech}[1]{\textcolor{techblue}{\texttt{#1}}}
\newcommand{\skillitem}[2]{\item[\ding{108}] \textbf{#1}: #2}

% ============ 超链接设置 ============
\hypersetup{
  colorlinks=true,
  urlcolor=techblue,
  linkcolor=primary,
  citecolor=secondary
}

\begin{document}

% ============ 个人信息 ============
\begin{minipage}{0.65\textwidth}
  {\Huge\bfseries\sffamily 张凯} \\[8pt]
  {\large\texttt{电话:+86 15122986177} | \texttt{邮箱:\href{mailto:15122986177@163.com}{15122986177@163.com}}} \\[6pt]
  {\large\texttt{上海 | 婚姻状况:已婚 | 出生日期:1992年3月11日}}
\end{minipage}
\hfill
\begin{minipage}{0.3\textwidth}
  \centering
  \includegraphics[width=0.9\textwidth,height=3.5cm,keepaspectratio]{profile_photo.jpg}
\end{minipage}

\vspace{10pt}
{\color{primary}\hrule}
\vspace{12pt}

% ============ 职业概述 ============
\section*{职业概述}

\normalsize
5年汽车电子软件开发与集成经验,擅长基于\tech{Classic AUTOSAR}工具链开发、\tech{RTE}接口配置与代码生成、软件集成与发布流程管理,擅长利用\tech{Python}提升集成测试环境构建和测试效率, 具备\tech{CI/CD}流水线设计经验, 拥有跨团队技术协调能力,熟悉发布计划对接、系统需求分派、软件版本与系统需求关系维护、市场质量问题对接分派及跟踪流程。

% ============ 关键能力 ============
\section*{关键能力}

\begin{itemize}
  \skillitem{AUTOSAR架构与软件集成}{精通\tech{Classic AUTOSAR}架构,熟练掌握\tech{RTE}接口连接、\tech{OS}时序挂载、\tech{RTE}代码生成,具备高性能中央计算平台和车身域控制器集成经验}
  \skillitem{C语言嵌入式开发}{熟练使用\tech{C}语言进行嵌入式软件开发,主导\tech{Auth}安全认证模块开发,实现\tech{BLE}、\tech{TBOX}、\tech{VCU}、\tech{ESCL}等部件的安全认证功能,基于\tech{UDS}诊断协议和\tech{ISO 14229}标准确保通信安全和系统稳定性}
  \skillitem{CI/CD与版本管理}{精通\tech{Jenkins}、\tech{Gerrit}、\tech{Git}、\tech{SVN}等工具,能够独立完成软件版本管理、构建、打包、发布全流程,具备处理集成冲突与发布异常的能力}
  \skillitem{需求管理与质量管控}{熟悉需求管理工具,具备系统需求分析、分派及跟踪能力,能够对软件质量问题进行初筛、分派和跟踪,建立软件版本与系统需求关系管理机制}
  \skillitem{跨团队协调与沟通}{具备良好的跨团队沟通能力,能协调开发、测试、系统等多个团队,作为技术接口人推动软件问题解决和质量管控,负责市场质量问题对接、分派及跟踪}
\end{itemize}

\subsection*{技术技能矩阵}

\begin{tabularx}{\textwidth}{@{}p{3.2cm}X@{}}
  \textbf{需求管理} & \tech{Doors} \\
  \textbf{问题追踪} & \tech{Jira} \\
  \textbf{编程语言} & \tech{Python}, \tech{C}, \tech{Matlab(Stateflow)} \\
  \textbf{AutoSAR工具链} & \tech{DaVinci Developer}, \tech{DaVinci Configurator}, \tech{CANoe}, \tech{vTESTstudio} \\
  \textbf{CI/CD工具} & \tech{Jenkins}, \tech{Gerrit}, \tech{Git}, \tech{SVN} \\
  \textbf{调试工具} & \tech{iSYSTEM winIDEA}, \tech{Lauterbach Trace32}
\end{tabularx}

% ============ 工作经历 ============
\section*{工作经历}

\subsection*{沃尔沃汽车(亚太)投资控股有限公司 Volvo Car (Asia Pacific) Investment Holding Co., Ltd.}
\textbf{中央计算平台集成主管工程师} \hfill \textit{2023年11月 -- 至今,中国上海}

\subsubsection*{SPA3/GPA高性能计算平台集成|AUTOSAR工具链开发负责人}
\hfill \textit{2023年11月 -- 2025年7月}

\begin{itemize}
  \item[\ding{51}] \highlight{AUTOSAR工具链开发与集成}:负责高性能计算平台的\tech{AUTOSAR CP}架构集成,主导基于\tech{Python}的\tech{yaml2arxml}转换工具和\tech{SWC}配置管理工具开发,实现从\tech{yaml}脚本配置文件到\tech{ARXML}和代码的全流程分布式管理自动化,替代\tech{Davinci Developer}核心功能,显著提升综合配置效率
  \item[\ding{51}] \highlight{RTE代码生成}:负责\tech{Contract Phase Generation}生成: \tech{SWC}的\tech{C}语言头文件和\tech{implementation template},以及RTE生成.
  \item[\ding{51}] \highlight{ARXML文件处理与合规检查}:负责\tech{ECUExtract}的合规检查和错误修复,确保AUTOSAR架构一致性和配置文件的完整性
  \item[\ding{51}] \highlight{软件集成与部署}:通过劳德巴赫调试器进行\tech{C}代码级调试,进行软件编译-烧录-测试,负责软件集成测试和系统级问题定位
  \item[\ding{51}] \highlight{整车测试问题跟踪和推动解决}:根据整车测试发现问题进行集成测试复现, 初步分析定位, 并推动相关团队解决问题, 及时交付新版本软件
\end{itemize}

\subsubsection*{CI/CD架构与工具链开发|流水线设计开发者}
\hfill \textit{2025年8月 -- 至今}

\begin{itemize}
  \item[\ding{51}] \highlight{CI/CD流水线架构与脚本开发}:设计企业级CI/CD流水线系统,基于\tech{Jenkins-Gerrit}技术栈实现代码质量门禁系统,开发流水线脚本并集成静态检查工具和编译器,确保代码质量合规
  \item[\ding{51}] \highlight{多进程Jenkins Job}:设计主从Job并发执行架构,使用多进程并发,提升CI/CD流水线执行效率\textbf{30\%}
  \item[\ding{51}] \highlight{可配置化开发}:设计\tech{YAML}配置驱动的测试框架,支持同一个\tech{Jenkins Job}的多版本\tech{pipeline}脚本的动态加载和执行,提升流水线开发的效率
\end{itemize}



\subsection*{科世达(上海)机电有限公司 KOSTAL (Shanghai) Mechatronic Co., Ltd.}
\textbf{车身域控制器开发工程师} \hfill \textit{2020年12月 -- 2023年11月,中国上海}

\subsubsection*{车身域控制器集成负责人}
\hfill \textit{2020年12月 -- 2023年11月}

\begin{itemize}
  \item[\ding{51}] \highlight{AUTOSAR架构集成}:负责车身域控制器的\tech{AUTOSAR CP}架构和集成,主导\tech{SWC}间\tech{RTE}接口设计和连接、\tech{OS}时序挂载和基础软件组件集成,确保软件架构和接口连接符合\tech{SWC}功能需求
  \item[\ding{51}] \highlight{软件集成与发布}:进行完整的编译、烧录、测试流程,负责软件集成测试和系统级问题定位,建立软件版本与需求关系管理,确保软件符合功能需求
  \item[\ding{51}] \highlight{质量管控与问题跟踪}:作为技术接口人协调\tech{SWC}开发、\tech{BSW}开发、集成测试、系统测试以及硬件等多个团队,推动产品问题解决,建立质量问题分派和跟踪机制,确保项目按时交付
  \item[\ding{51}] \highlight{OEM整车测试问题跟踪和推动解决}:车身域控制器交付OEM后, 根据OEM的整车测试发现问题进行集成测试复现, 初步分析定位, 并推动相关团队解决问题, 及时交付新版本软件
  \item[\ding{51}] \highlight{自动化测试工具开发}:基于\tech{Python}开发自动化测试脚本生成工具,构建自动化测试框架提升集成测试效率和质量保证能力
  \item[\ding{51}] \highlight{部分SWC开发}:负责车身域控制器的部分\tech{SWC}模块开发,包括\tech{Auth}安全认证模块(基于\tech{C}语言), \tech{NvmManager}模块(基于\tech{C}语言), \tech{IoHwAb}模块(基于\tech{C}语言), 内灯模块(基于\tech{Matlab(Stateflow)})
\end{itemize}

% ============ 教育背景 ============
\section*{教育背景}

\begin{itemize}
  \item \textbf{德国卡尔斯鲁厄理工学院(KIT)} \hfill
    \textit{机电一体化及信息技术硕士} \hfill
    \textit{2017.04 -- 2020.10}
  \begin{itemize}
    \item 深化方向:工业自动化、机器人技术
    \item 优秀毕业设计:基于\tech{Python}和深度学习的动态眼球追踪系统数据质量优化
  \end{itemize}

  \item \textbf{河北工业大学(211)} \hfill
    \textit{机械设计制造及其自动化学士} \hfill
    \textit{2012年9月 -- 2016年7月}
  \begin{itemize}
    \item 优秀毕业设计:人体工程学自动调节座椅
  \end{itemize}
\end{itemize}

% ============ 证书与其他 ============
\section*{证书与其他}

\begin{itemize}
  \item \textbf{外语能力}:英语(雅思:6.5),德语(TestDaf:16/C1)
  \item \textbf{科世达2022年度优秀员工}
  \item \textbf{科世达AE电子开发部年会主持人}
\end{itemize}

\end{document}
