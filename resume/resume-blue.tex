\documentclass[11pt,a4paper]{article}

% 使用ctex包支持中文
\usepackage{ctex}
\usepackage{geometry}
\geometry{left=0.75in,right=0.75in,top=0.75in,bottom=0.75in}
\usepackage{hyperref}
\usepackage{titlesec}
\usepackage{enumitem}
\usepackage{fancyhdr}
\usepackage{xcolor}
\usepackage{tikz}

% 定义专业配色方案
\definecolor{mainblue}{RGB}{0, 51, 102}        % 深蓝色 - 主色调
\definecolor{lightblue}{RGB}{51, 102, 153}     % 中蓝色
\definecolor{accentblue}{RGB}{102, 153, 204}   % 浅蓝色 - 强调色
\definecolor{bgblue}{RGB}{240, 248, 255}       % 极浅蓝背景
\definecolor{textdark}{RGB}{51, 51, 51}        % 深灰色文字

% 配置hyperref颜色
\hypersetup{
    colorlinks=true,
    urlcolor=lightblue,
    linkcolor=mainblue
}

% 标题格式设置 - 优雅蓝色系
\titleformat{\section}{\Large\bfseries\color{mainblue}\raggedright}{}{0em}{}[{\color{accentblue}\titlerule}]
\titlespacing{\section}{0pt}{3pt}{3pt}

\titleformat{\subsection}{\large\bfseries\color{lightblue}\raggedright}{}{0em}{}
\titlespacing{\subsection}{0pt}{2pt}{2pt}

% 项目符号设置
\setlist{leftmargin=0.25in,itemsep=2pt,parsep=0pt,topsep=2pt}

% 页面样式
\pagestyle{fancy}
\fancyhf{}
\renewcommand{\headrulewidth}{0pt}
\renewcommand{\footrulewidth}{0pt}

\begin{document}

% 个人信息卡片 - 优雅设计
\begin{tikzpicture}[remember picture,overlay]
    \fill[mainblue] (current page.north west) rectangle ([yshift=-2.5cm]current page.north east);
\end{tikzpicture}

\vspace{1.5cm}
\begin{center}
    {\Huge\bfseries\color{white}张凯}\\[0.3em]
    {\Large\color{white}软件集成工程师}\\[0.8em]
    {\normalsize\color{white}
        \texttt{+86 15122986177} $\bullet$
        \texttt{15122986177@163.com} $\bullet$
        1992年3月11日
    }
\end{center}

\vspace{0.5cm}

% 教育背景
\section*{教育背景}

\textbf{\color{mainblue}卡尔斯鲁厄理工学院 (KIT)} \hfill \textit{2017年4月 -- 2020年10月}\\
\textit{机电一体化与信息技术硕士} \hfill \textit{德国卡尔斯鲁厄}\\
\begin{itemize}
    \item 专业方向1:工业自动化
    \item 专业方向2:机器人技术
\end{itemize}

\textbf{\color{mainblue}河北工业大学} \hfill \textit{2012年9月 -- 2016年7月}\\
\textit{机械设计制造及其自动化学士} \hfill \textit{中国天津}\\
\begin{itemize}
    \item 毕业论文:人性化自动调节椅的设计
\end{itemize}

% 工作经历
\section*{工作经历}

\textbf{\color{mainblue}沃尔沃汽车 | 软件集成工程师} \hfill \textit{2023年11月 -- 至今}\\
\textit{中国上海}

\textbf{\color{lightblue}SPA3/GPA HI Integration项目 | AUTOSAR系统集成与工具开发} \hfill \textit{2023年11月 -- 2025年7月}
\begin{itemize}
    \item \textbf{负责跨域融合的软件架构集成},主导ZC(NVIDIA Orin)与HI(Infineon TC397)的系统集成工作
    \item \textbf{开发AUTOSAR集成工具链},包括stakeholder\_build\_template构建模板和自动化代码生成工具
    \item \textbf{管理SWC配置和代码生成},自动生成ECUextract和.c/.h框架代码,提升开发效率
    \item \textbf{开发配置转换工具},实现yaml2arxml自动转换,简化配置管理流程
    \item \textbf{替代Davinci Developer工具},开发定制化工具解决原有工具链的局限性
    \item \textbf{SOME/IP服务开发与测试},负责车载以太网通信服务的设计、配置和验证
    \item \textbf{车载网络测试},使用CANoe配合VN5650设备进行通信协议测试和调试
\end{itemize}

\textbf{\color{lightblue}Java APP代码门禁Pipeline架构设计} \hfill \textit{2025年8月 -- 至今}
\begin{itemize}
    \item \textbf{主导CI/CD架构设计},基于Jenkins和Groovy脚本构建代码质量检查和自动化测试系统
    \item \textbf{实现主从Job并发架构},结合ktlint、detekt进行静态代码检查,提升代码审查效率
    \item \textbf{设计分布式数据管理},使用MongoDB实现检查结果的分批记录和集中仲裁机制
    \item \textbf{构建可配置测试框架},支持YAML配置的自动化测试触发,提升测试灵活性
\end{itemize}

\textbf{\color{mainblue}科世达(上海)机电有限公司 | 集成开发负责人/软件工程师} \hfill \textit{2020年12月 -- 2023年11月}\\
\textit{中国上海}

\textbf{\color{lightblue}车身域控制器(CEM)项目 | 集成开发负责人} \hfill \textit{2022年12月 -- 2023年11月}
\begin{itemize}
    \item 搭建集成测试的硬件和软件环境,编译软件、烧录硬件并进行集成测试
    \item 设计生产线最终检验程序,设计EMC上位机和PCBA的EMC测试专用软件
    \item 定位集成测试和系统测试中的问题,与相关工程师沟通改进软件
    \item 定位客户问题并帮助相关工程师改进软件,向客户发布软件并提供维护服务
\end{itemize}

\textbf{\color{lightblue}Vector工具链的自动化集成测试平台 | 创建者、推广者} \hfill \textit{2023年1月 -- 2023年6月}
\begin{itemize}
    \item 基于客户的CAN/LIN/IO矩阵使用Python生成CANoe CAPL脚本和vTESTstudio配置文件
    \item 在vTESTstudio中构建自动化测试项目,根据客户功能规格编写自动化测试用例
    \item 将自动化测试用例导入CANoe进行自动化测试,提升测试效率
\end{itemize}

\textbf{\color{lightblue}客户项目经验}
\begin{itemize}
    \item \textbf{宇通客车} - KBCM(车身控制器)项目 | 内饰灯模块软件开发工程师 (2022年8月 -- 2023年8月)
    \item \textbf{长城汽车} - 欧拉好猫系列KBCM项目 | 集成开发负责人 (2022年2月 -- 2023年8月)
    \item \textbf{江西五十铃} - PEPS(无钥匙进入系统) \& ESCL(电子转向锁)项目 | 集成开发负责人 (2022年3月 -- 2023年3月)
    \item \textbf{理想汽车} - 车和家系列KBCM项目 | 集成开发负责人 (2022年3月 -- 2022年10月)
    \item \textbf{长城汽车} - 欧拉黑猫系列KBCM项目 | 集成开发负责人 | BLE/TBOX/VCU/ESCL认证模块软件工程师 (2021年1月 -- 2021年12月)
\end{itemize}

% 专业技能
\section*{专业技能}

\begin{itemize}
    \item \textbf{\color{mainblue}AUTOSAR架构与工具链:}熟悉Adaptive Platform和Classic Platform架构,精通Davinci Developer、Davinci Configurator
    \item \textbf{\color{mainblue}车载通信协议:}具备SOME/IP服务设计、配置和测试的完整项目经验,熟悉车载以太网架构
    \item \textbf{\color{mainblue}编程语言:}熟悉现代C++特性,精通Java开发,熟练使用Python进行工具开发
    \item \textbf{\color{mainblue}系统集成与自动化:}主导设计Jenkins-based的CI/CD系统,具备定制化开发工具的经验
\end{itemize}

% 特殊经历与荣誉
\section*{特殊经历与荣誉}

\begin{itemize}
    \item \textbf{\color{accentblue}2022年1月 -- 2022年12月:}获得科世达(上海)机电有限公司2022年度优秀员工奖
    \item \textbf{\color{accentblue}2022年3月 -- 2022年6月:}上海疫情封控期间驻公司协助完成多项紧急任务
    \item \textbf{\color{mainblue}2023年2月16日:}主持科世亚亚洲总部AE开发部中国新年晚宴
    \item \textbf{\color{mainblue}2022年2月28日:}主持科世亚亚洲总部AE开发部年度表彰大会
\end{itemize}

\end{document}