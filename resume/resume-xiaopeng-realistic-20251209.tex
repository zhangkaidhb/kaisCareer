\documentclass[11pt,a4paper]{article}
\usepackage{ctex}
\usepackage{geometry}
\geometry{left=2cm,right=2cm,top=2cm,bottom=2cm}
\usepackage{titlesec}
\usepackage{enumitem}
\usepackage{fancyhdr}
\usepackage{hyperref}

% 页面设置
\pagestyle{fancy}
\fancyhf{}
\renewcommand{\headrulewidth}{0pt}
\renewcommand{\footrulewidth}{0pt}

% 标题格式
\titleformat{\section}
  {\Large\bfseries\sffamily\raggedright}
  {}
  {0em}
  {}
  [{\titlerule[0.5pt]}]

\titleformat{\subsection}
  {\large\bfseries\sffamily}
  {}
  {0em}
  {}

\titlespacing{\section}{0pt}{10pt}{6pt}
\titlespacing{\subsection}{0pt}{6pt}{4pt}

% 列表格式
\setlist{
  leftmargin=0pt,
  itemsep=3pt,
  parsep=0pt,
  topsep=3pt,
  labelsep=8pt
}

% 超链接设置
\hypersetup{
  colorlinks=true,
  urlcolor=blue,
  linkcolor=black
}

\begin{document}

% 个人信息
\begin{center}
  {\Huge\bfseries\sffamily 张凯} \\[6pt]
  {\large\texttt{Tel.: +86 15122986177} | \texttt{E-Mail: 15122986177@163.com}} \\[6pt]
  {\large 个人简历} \\[3pt]
  {\normalsize 基本信息:张凯 | 婚姻状况:已婚 | 出生日期:1992年3月11日 | 毕业时间:2020年10月}
\end{center}

\vspace{12pt}

\hrule
\vspace{12pt}

% 核心技能摘要
\section*{核心技能摘要}

\noindent
\textbf{\sffamily 确定性中间件开发工程师 | AUTOSAR架构专家} \\
5年汽车电子软件开发经验,专注于跨域融合软件架构设计、确定性中间件开发、实时通信系统。精通AUTOSAR AP/CP架构、TSN网络调度、车载以太网通信中间件,具备从架构设计到实际部署的全栈开发能力。

\vspace{6pt}

\textbf{核心能力矩阵:}
\begin{itemize}
  \item \textbf{AUTOSAR架构}:精通AUTOSAR CP标准架构,熟悉AP架构协议及ARA服务,具备基于AUTOSAR CP的车身控制器架构经验
  \item \textbf{系统集成经验}:具备高性能计算平台集成经验,熟悉车身域控制器集成开发,掌握车载网络通信架构(CAN/LIN/以太网)和系统级集成测试
  \item \textbf{确定性通信中间件}:主导SOME/IP服务配置测试,深入理解TSN协议族(IEEE802.1AS gPTP、802.1Qbv、802.1Qav)及QoS保障机制,具备车载以太网通信经验,了解DDS通信协议原理
  \item \textbf{软件开发能力}:精通Python面向对象编程及设计模式,具备多线程开发经验;精通MatLab Stateflow开发,具备丰富的汽车电子SWC开发经验;熟悉C++编程,具备车载通信系统调试经验
  \item \textbf{工具链与自动化}:精通Vector工具链(DaVinci Developer/Configurator、CANoe、vTESTstudio),具备自动化脚本开发与CI/CD流水线设计开发经验,熟悉Linux系统环境
\end{itemize}

% 教育信息
\section*{教育信息}

\noindent
\textbf{\sffamily 卡尔斯鲁厄理工学院(Karlsruhe Institut für Technologie)(KIT)} \hfill
\textit{德国工学硕士} \\
\textbf{2017年4月 – 2020年10月} \\[4pt]
\textbf{专业:机电一体化及信息技术} \\[4pt]
\begin{itemize}
  \item 深化方向1:工业自动化
  \item 深化方向2:机器人技术
  \item 毕业设计:用于优化动态眼球追踪系统数据质量的分析方法的设计和验证
\end{itemize}

\vspace{8pt}

\noindent
\textbf{\sffamily 河北工业大学(211)} \hfill
\textit{中国工学学士} \\
\textbf{2012年9月 – 2016年7月} \\[4pt]
\textbf{本专业:机械设计制造及其自动化} \\[4pt]
\begin{itemize}
  \item 毕业设计:人体工程学自动调节座椅(优秀毕业设计)
\end{itemize}

% 实践经验
\section*{实践经验}

\noindent
\textbf{\sffamily 沃尔沃汽车(亚太)投资控股有限公司 Volvo Car (Asia Pacific) Investment Holding Co., Ltd.} \hfill
\textit{中国上海} \\
\textbf{2023年11月 – 至今}

\vspace{6pt}
\textbf{\large SPA3/GPA高性能计算平台集成 | AUTOSAR工具链开发负责人} \hfill
\textit{2023年11月 – 2025年7月}
\begin{itemize}
  \item \textbf{AUTOSAR工具链开发与集成}:负责高性能计算平台的AUTOSAR CP架构集成,主导yaml2arxml转换工具和SWC配置管理工具开发,实现从yaml脚本配置文件到ARXML和代码的全流程分布式管理自动化,替代Davinci Developer核心功能,显著提升综合配置效率
  \item \textbf{ARXML文件处理与合规检查}:负责ECUExtract的合规检查和错误修复,确保AUTOSAR架构一致性和配置文件的完整性
  \item \textbf{车载以太网中间件配置}:主导SOME/IP服务在Davinci Configurator中的完整配置流程,包含SoAd、SD、BswM、EcuC、ComM、PduR等模块配置,负责EthTSyn模块的IEEE802.1AS(gPTP)时间同步配置,确保服务正确部署和确定性通信保障
  \item \textbf{通信验证与测试}:使用CANoe配合VN5650进行SOME/IP服务通信测试,验证车载以太网通信机制和时序要求

  \item \textbf{系统调试与部署}:通过Linux环境配置和劳德巴赫调试器进行C代码级调试,保障系统集成稳定性
\end{itemize}

\vspace{6pt}
\textbf{\large CI/CD架构与工具链开发 | 流水线设计负责人} \hfill
\textit{2025年8月 – 至今}
\begin{itemize}
  \item \textbf{CI/CD流水线架构与脚本开发}:支持设计企业级CI/CD流水线系统,基于Gerrit-Jenkins-Gradle技术栈实现代码质量门禁系统,使用Groovy语言开发Jenkins Job脚本并集成ktlint、detekt等静态检查工具和Gradle编译器,确保代码质量合规
  \item \textbf{并发架构设计}:设计主从Job并发执行架构,使用多进程并发,提升CI/CD流水线执行效率30\%
  \item \textbf{数据管理方案}:基于数据库服务设计检查结果存储和仲裁机制,实现基于RPC实现的跨服务器代码检查结果分批记录与集中仲裁,提升门禁系统的灵活性
  \item \textbf{可配置化开发}:设计YAML配置驱动的测试框架,支持同一个Jenkins Job的多版本pipeline脚本的动态加载和执行
\end{itemize}

\vspace{8pt}

\noindent
\textbf{\sffamily 科世达(上海)机电有限公司,KOSTAL (Shanghai) Mechatronic Co., Ltd.} \hfill
\textit{中国上海} \\
\textbf{2020年12月 – 2023年11月}

\vspace{6pt}
\textbf{\large 科世达车身域控制器CEM项目 | AUTOSAR集成与中间件开发负责人} \hfill
\textit{2022年12月 – 2023年11月}
\begin{itemize}
  \item \textbf{AUTOSAR架构集成}:负责车身域控制器的AUTOSAR CP架构集成,主导SWC间通信配置和中间件组件集成
  \item \textbf{系统级调试与部署}:建立完整的编译-烧录-测试流程,负责软件集成测试和系统级问题定位,确保系统稳定性
  \item \textbf{测试工具链开发}:设计产线终检程序和EMC测试上位机软件,构建自动化测试框架提升测试效率
  \item \textbf{跨团队技术协调}:作为技术接口人推动软件问题解决,确保项目按时交付和后期维护支持
\end{itemize}

\vspace{6pt}
\textbf{\large 车载网络自动化测试平台开发 | Vector工具链专家} \hfill
\textit{2023年1月 – 2023年6月}
\begin{itemize}
  \item \textbf{自动化工具链开发}:基于Python开发自动化测试脚本生成工具,根据客户通信矩阵生成CANoe CAPL脚本和vTESTstudio配置
  \item \textbf{通信协议测试验证}:针对CAN/LIN/以太网通信协议设计测试用例,验证车载网络通信的可靠性和实时性
  \item \textbf{测试工程构建}:构建完整的vTESTstudio自动化测试工程,实现从测试用例设计到执行的全流程自动化
  \item \textbf{测试技术推广}:在团队内推广自动化测试方法论,提升团队测试效率和质量保证能力
\end{itemize}

\vspace{8pt}

\textbf{\large AUTOSAR车身控制器项目经验}
\begin{itemize}
  \item \textbf{长城汽车} - 欧拉好猫/黑猫系列KBCM项目 | 车身域控制器集成负责人 \hfill \textit{2021年1月 – 2023年8月}
  \begin{itemize}
    \item 主导AUTOSAR CP架构下的车身控制器集成开发,负责BLE/TBOX/VCU/ESCL等安全关键模块的通信集成,符合ISO 26262功能安全要求
    \item 建立安全关键系统的通信测试验证体系,确保车身域控制器在ASIL等级要求下的功能安全性
  \end{itemize}
  \item \textbf{理想汽车} - 车和家系列KBCM项目 | 集成开发负责人 \hfill \textit{2022年3月 – 2022年10月}
  \begin{itemize}
    \item 负责新能源汽车车身控制器的AUTOSAR架构集成和功能安全验证
  \end{itemize}
  \item \textbf{江西五十铃} - PEPS无钥匙进入系统 | 集成开发负责人 \hfill \textit{2022年3月 – 2023年3月}
  \begin{itemize}
    \item 主导车辆防盗安全系统的通信协议集成和测试验证,确保PEPS系统满足ISO 26262 ASIL-B功能安全等级要求
  \end{itemize}
  \item \textbf{宇通客车} - KBCM车身控制器项目 | 软件开发工程师 \hfill \textit{2022年8月 – 2023年8月}
  \begin{itemize}
    \item 负责商用车车身控制器的内灯模块开发和系统集成
  \end{itemize}
\end{itemize}

% 毕业设计
\section*{毕业设计}

\noindent
\textbf{\sffamily 2019年8月 – 2020年8月} \\
\textbf{Institut für Arbeitswissenschaft und Betriebsorganisation(KIT),德国} \\
\textbf{项目:用于优化动态眼球追踪系统数据质量的分析方法的设计和验证}
\begin{itemize}
  \item 设计实验收集眼球追踪系统的点云追踪图像数据
  \item 利用开源目标识别AI模型Mask\_RCNN将图像数据转化为坐标数据集
  \item 在Python中利用误差空间插值模型补偿眼球追踪系统,从而实现优化眼球追踪系统数据质量的目标
\end{itemize}

\vspace{12pt}

% 技能与爱好
\section*{技能与爱好}

\textbf{外语:}
\begin{itemize}
  \item 英语(雅思:6.5)
  \item 德语(TestDaf:16/C1)
\end{itemize}

\textbf{软件工具技能:}
\begin{itemize}
  \item \textbf{AI工具与IDE:} 精通Claude Code、Cursor等先进AI IDE,熟悉主流LLM模型优势(Claude、GPT、Gemini、Deepseek、GLM等)
  \item \textbf{编程语言:} Python、Groovy、C++、MATLAB(Stateflow)
  \item \textbf{Vector工具链:} DaVinci Developer\&Configurator、CANoe、vTESTstudio
  \item \textbf{CI/CD工具:} Jenkins、Gerrit、Gradle、MongoDB、Groovy脚本、Git、Smart SVN
  \item \textbf{代码工具:} SourceInsight、Eclipse、VS Code、Pycharm
  \item \textbf{调试工具:} iSYSTEM winIDEA、Lauterbach Trace32
  \item \textbf{通讯工具:} 飞书、Teams、Zoom
  
\end{itemize}

\textbf{爱好特长:}
\begin{itemize}
  \item 中长跑、硬笔书法、无动力帆船
\end{itemize}

% 特别经历和荣誉
\section*{特别经历和荣誉}

\begin{itemize}
  \item \textbf{科世达亚洲总部AE电子开发部年夜饭晚会主持人} \hfill \textit{2023年2月16日}
  \item \textbf{科世达(上海)机电有限公司2022年度优秀员工} \hfill \textit{2022年1月 – 2022年12月}
  \item \textbf{科世达亚洲总部AE电子开发部年度表彰大会主持人} \hfill \textit{2022年2月28日}
  \item \textbf{上海市防疫封控期间驻守公司,协助完成多项项目紧急任务} \hfill \textit{2022年3月28日 – 2022年6月1日}
\end{itemize}

\end{document}