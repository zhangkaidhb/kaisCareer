\documentclass[11pt,a4paper]{article}
\usepackage{ctex}
\usepackage{geometry}
\geometry{left=2cm,right=2cm,top=2cm,bottom=2cm}
\usepackage{titlesec}
\usepackage{enumitem}
\usepackage{fancyhdr}
\usepackage{hyperref}

% 页面设置
\pagestyle{fancy}
\fancyhf{}
\renewcommand{\headrulewidth}{0pt}
\renewcommand{\footrulewidth}{0pt}

% 标题格式
\titleformat{\section}
  {\Large\bfseries\sffamily\raggedright}
  {}
  {0em}
  {}
  [{\titlerule[0.5pt]}]

\titleformat{\subsection}
  {\large\bfseries\sffamily}
  {}
  {0em}
  {}

\titlespacing{\section}{0pt}{10pt}{6pt}
\titlespacing{\subsection}{0pt}{6pt}{4pt}

% 列表格式
\setlist{
  leftmargin=0pt,
  itemsep=3pt,
  parsep=0pt,
  topsep=3pt,
  labelsep=8pt
}

% 超链接设置
\hypersetup{
  colorlinks=true,
  urlcolor=blue,
  linkcolor=black
}

\begin{document}

% 个人信息
\begin{center}
  {\Huge\bfseries\sffamily 张凯} \\[6pt]
  {\large\texttt{Tel.: +86 15122986177} | \texttt{E-Mail: 15122986177@163.com}} \\[6pt]
  {\large 个人简历} \\[3pt]
  {\normalsize 基本信息:张凯 | 婚姻状况:已婚 | 出生日期:1992年3月11日 | 毕业时间:2020年10月}
\end{center}

\vspace{12pt}

\hrule
\vspace{12pt}

% 核心技能摘要
\section*{核心技能摘要}

\noindent
\textbf{\sffamily 嵌入式软件与工具链开发工程师} \\
5年汽车电子软件开发经验,专注于AUTOSAR工具链应用、SOME/IP服务配置测试、CI/CD系统开发。熟练掌握Vector工具链、Python自动化开发、C++编程、系统架构设计。

% 教育信息
\section*{教育信息}

\noindent
\textbf{\sffamily 卡尔斯鲁厄理工学院(Karlsruhe Institut für Technologie)(KIT)} \hfill
\textit{德国工学硕士} \\
\textbf{2017年4月 – 2020年10月} \\[4pt]
\textbf{专业:机电一体化及信息技术} \\[4pt]
\begin{itemize}
  \item 深化方向1:工业自动化
  \item 深化方向2:机器人技术
  \item 毕业设计:用于优化动态眼球追踪系统数据质量的分析方法的设计和验证
\end{itemize}

\vspace{8pt}

\noindent
\textbf{\sffamily 河北工业大学(211)} \hfill
\textit{中国工学学士} \\
\textbf{2012年9月 – 2016年7月} \\[4pt]
\textbf{本专业:机械设计制造及其自动化} \\[4pt]
\begin{itemize}
  \item 毕业设计:人体工程学自动调节座椅(优秀毕业设计)
\end{itemize}

% 实践经验
\section*{实践经验}

\noindent
\textbf{\sffamily 沃尔沃汽车(Volvo Cars)} \hfill
\textit{中国上海} \\
\textbf{2023年11月 – 至今}

\vspace{6pt}
\textbf{\large SPA3/GPA HI Integration | 集成工具开发负责人} \hfill
\textit{2023年11月 – 2025年7月}
\begin{itemize}
  \item 开发stakeholder\_build\_template工具,管理各SWC配置,自动生成ECUextract和.c/.h框架代码
  \item 创建yaml2arxml工具,实现配置文件到AUTOSAR架构的自动转换,有效替代Davinci Developer部分功能
  \item 负责ECUExtract\_Unflattened.arxml文件检查,解决Davinci Developer和Configurator配置错误
  \item 进行SOME/IP服务在Davinci Configurator中的配置工作,确保服务正确部署
  \item 使用CANoe配合VN5650设备进行SOME/IP服务的通信测试和验证
  \item 通过PuTTy访问Linux系统进行环境配置,使用劳德巴赫进行C代码断点调试
\end{itemize}

\vspace{6pt}
\textbf{\large CI/CD Pipeline架构设计师} \hfill
\textit{2025年8月 – 至今}
\begin{itemize}
  \item 设计并实施Java APP代码门禁pipeline系统:业务仓通过Git上传Gerrit触发Jenkins Job进行代码检查
  \item 基于Groovy脚本开发Jenkins Job,集成Gradle的ktlint、detekt进行静态代码检查
  \item 设计主从Job架构实现静态代码检查和Build的并发执行,提升CI/CD效率
  \item 使用MongoDB数据库实现检查结果的分批记录和集中仲裁机制
  \item 构建CI-Pipeline脚本自动化测试系统,解决共享库动态加载问题
  \item 设计可配置测试框架yaml,实现依据不同Trigger来源拉取不同版本的pipeline脚本
\end{itemize}

\vspace{8pt}

\noindent
\textbf{\sffamily 科世达(上海)机电有限公司,KOSTAL (Shanghai) Mechatronic Co., Ltd.} \hfill
\textit{中国上海} \\
\textbf{2020年12月 – 2023年11月}

\vspace{6pt}
\textbf{\large 科世达车身域控制器 CEM 项目 | 集成开发负责人} \hfill
\textit{2022年12月 – 2023年11月}
\textbf{集成开发负责人职责:}
\begin{itemize}
  \item 集成测试软硬件环境的搭建
  \item 编译软件,烧入硬件,集成测试
  \item 设计科世达产线终检程序
  \item 设计硬件所需EMC上位机和EMC测试软件
  \item 定位集成测试和系统测试出现的问题,推动相关工程师完善软件
  \item 定位客户端出现的问题,推动相关工程师完善软件
  \item 软件放行给客户,以及后期的维护
\end{itemize}

\vspace{6pt}
\textbf{\large 基于 Vector 工具链的车身控制器自动化集成测试平台 | 创造者、推广者} \hfill
\textit{2023年1月 – 2023年6月}
\begin{itemize}
  \item 利用Python基于客户CAN/LIN/IO Matrix生成CANoe CAPL脚本和vTESTstudio配置文件
  \item 构建vTESTstudio自动化测试工程
  \item 基于客户功能规范编写自动化测试用例
  \item 将自动化测试用例导入到CANoe中进行自动化测试
\end{itemize}

\vspace{8pt}

\textbf{\large 客户项目经验}
\begin{itemize}
  \item \textbf{宇通客车股份有限公司} - KBCM(车身控制器)项目 | 内灯模块软件开发工程师 \hfill \textit{2022年8月 – 2023年8月}
  \item \textbf{长城汽车股份有限公司} - 欧拉好猫系列KBCM(车身控制器)项目 \hfill \textit{2022年2月 – 2023年8月}
  \item \textbf{江西五十铃皮卡系列 PEPS(无钥匙进入系统)项目} | 集成开发负责人 \hfill \textit{2022年3月 – 2023年3月}
  \item \textbf{江西五十铃皮卡系列 ESCL(电子转向柱锁)项目} | 集成开发负责人 \hfill \textit{2022年3月 – 2023年3月}
  \item \textbf{理想汽车} - 车和家系列KBCM(车身控制器)项目 | 集成开发负责人 \hfill \textit{2022年3月 – 2022年10月}
  \item \textbf{长城汽车股份有限公司} - 欧拉黑猫系列KBCM(车身控制器)项目 \hfill \textit{2021年1月 – 2021年12月} \\
    \hspace{2cm}集成开发负责人 | BLE/TBOX/VCU/ESCL认证模块软件开发工程师
\end{itemize}

% 毕业设计
\section*{毕业设计}

\noindent
\textbf{\sffamily 2019年8月 – 2020年8月} \\
\textbf{Institut für Arbeitswissenschaft und Betriebsorganisation(KIT),德国} \\
\textbf{项目:用于优化动态眼球追踪系统数据质量的分析方法的设计和验证}
\begin{itemize}
  \item 设计实验收集眼球追踪系统的点云追踪图像数据
  \item 利用开源目标识别AI模型Mask\_RCNN将图像数据转化为坐标数据集
  \item 在Python中利用误差空间插值模型补偿眼球追踪系统,从而实现优化眼球追踪系统数据质量的目标
\end{itemize}

\vspace{8pt}

\noindent
\textbf{\sffamily 2018年10月 – 2019年1月} \\
\textbf{系统控制实践,Institut für Regelungs- und Steuerungssysteme (KIT) 德国}
\begin{itemize}
  \item \textbf{倒立摆控制}
  \begin{itemize}
    \item 利用状态空间法构建系统模型
    \item 在MATLAB/SIMULINK中进行系统控制模拟
    \item 建立观测器并用最小二乘控制器实现平衡控制
  \end{itemize}
  \item \textbf{后轴试验台控制}
  \begin{itemize}
    \item 利用状态空间法构建系统模型
    \item 对系统极点补偿,并进行解耦控制
    \item 通过传递函数给定PID控制器各项参数
    \item 在MATLAB/SIMULINK中模拟运行并进行PID参数整定
  \end{itemize}
\end{itemize}

\vspace{8pt}

\noindent
\textbf{\sffamily 2015年5月 – 2015年7月} \\
\textbf{PLC导轨滑块加工线设计 | 河北工业大学}
\begin{itemize}
  \item 根据载荷要求和所选分布结构计算所需导轨和滑块的功率和载荷
  \item 利用计算所得功率和在要求对西门子导轨和滑块进行选型
  \item 将厂家提供3D模型导入SolidWorks中进行模型装配
  \item 利用Adams对关键节点进行力学分析
  \item 编写PLC程序满足加工流程要求
\end{itemize}

\vspace{8pt}

\noindent
\textbf{\sffamily 2016年1月 – 2016年6月} \\
\textbf{本科毕业设计,河北工业大学 | 天津}
\textbf{项目:人体工程学自动调节座椅(原创题目)}
\begin{itemize}
  \item 结合人体工程学设计座椅结构
  \item 利用机构学设计座椅调节方式
  \item 在SolidWorks中1:1构建座椅3D模型
  \item 在AutoCAD中绘制2D装配图和2D零件加工图
\end{itemize}

% 技能与爱好
\section*{技能与爱好}

\textbf{外语:}
\begin{itemize}
  \item 英语(雅思:6.5)
  \item 德语(TestDaf:16/C1)
\end{itemize}

\textbf{软件工具技能:}
\begin{itemize}
  \item \textbf{编程语言:} Python、Java、Groovy、C++、MATLAB(Stateflow)
  \item \textbf{Vector工具链:} DaVinci Developer\&Configurator、CANoe、vTESTstudio、ECUExtract
  \item \textbf{CI/CD工具:} Jenkins、Gerrit、Gradle、MongoDB、Groovy脚本
  \item \textbf{代码工具:} Smart SVN、SourceInsight、Eclipse
  \item \textbf{调试工具:} iSYSTEM winIDEA、Lauterbach Trace32、劳德巴赫
  \item \textbf{生产力工具:} Xmind
\end{itemize}

\textbf{爱好特长:}
\begin{itemize}
  \item 台球、中长跑、硬笔书法、无动力帆船
\end{itemize}

% 特别经历和荣誉
\section*{特别经历和荣誉}

\begin{itemize}
  \item \textbf{科世达亚洲总部AE电子开发部年夜饭晚会主持人} \hfill \textit{2023年2月16日}
  \item \textbf{科世达(上海)机电有限公司2022年度优秀员工} \hfill \textit{2022年1月 – 2022年12月}
  \item \textbf{科世达亚洲总部AE电子开发部年度表彰大会主持人} \hfill \textit{2022年2月28日}
  \item \textbf{上海市防疫封控期间驻守公司,协助完成多项项目紧急任务} \hfill \textit{2022年3月28日 – 2022年6月1日}
\end{itemize}

\end{document}