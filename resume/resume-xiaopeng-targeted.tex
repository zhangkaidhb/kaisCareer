\documentclass[11pt,a4paper]{article}
\usepackage{ctex}
\usepackage{geometry}
\geometry{left=2cm,right=2cm,top=2cm,bottom=2cm}
\usepackage{titlesec}
\usepackage{enumitem}
\usepackage{fancyhdr}
\usepackage{hyperref}

% 页面设置
\pagestyle{fancy}
\fancyhf{}
\renewcommand{\headrulewidth}{0pt}
\renewcommand{\footrulewidth}{0pt}

% 标题格式
\titleformat{\section}
  {\Large\bfseries\sffamily\raggedright}
  {}
  {0em}
  {}
  [{\titlerule[0.5pt]}]

\titleformat{\subsection}
  {\large\bfseries\sffamily}
  {}
  {0em}
  {}

\titlespacing{\section}{0pt}{10pt}{6pt}
\titlespacing{\subsection}{0pt}{6pt}{4pt}

% 列表格式
\setlist{
  leftmargin=0pt,
  itemsep=3pt,
  parsep=0pt,
  topsep=3pt,
  labelsep=8pt
}

% 超链接设置
\hypersetup{
  colorlinks=true,
  urlcolor=blue,
  linkcolor=black
}

\begin{document}

% 个人信息
\begin{center}
  {\Huge\bfseries\sffamily 张凯} \\[6pt]
  {\large\texttt{Tel.: +86 15122986177} | \texttt{E-Mail: 15122986177@163.com}} \\[6pt]
  {\large 个人简历} \\[3pt]
  {\normalsize 基本信息:张凯 | 婚姻状况:已婚 | 出生日期:1992年3月11日 | 毕业时间:2020年10月}
\end{center}

\vspace{12pt}

\hrule
\vspace{12pt}

% 核心技能摘要
\section*{核心技能摘要}

\noindent
\textbf{\sffamily AUTOSAR AP/CP架构与确定性通信专家} \\
5年+汽车电子软件开发经验,专注于AUTOSAR架构开发、确定性中间件设计、跨域融合通信架构。精通C++11/14/17、Vector工具链、SOME/IP协议栈、实时系统调度算法。

% 教育信息
\section*{教育信息}

\noindent
\textbf{\sffamily 卡尔斯鲁厄理工学院(Karlsruhe Institut für Technologie)(KIT)} \hfill
\textit{德国工学硕士} \\
\textbf{2017年4月 – 2020年10月} \\[4pt]
\textbf{专业:机电一体化及信息技术} \\[4pt]
\begin{itemize}
  \item 深化方向1:工业自动化
  \item 深化方向2:机器人技术
  \item 毕业设计:用于优化动态眼球追踪系统数据质量的分析方法的设计和验证
\end{itemize}

\vspace{8pt}

\noindent
\textbf{\sffamily 河北工业大学(211)} \hfill
\textit{中国工学学士} \\
\textbf{2012年9月 – 2016年7月} \\[4pt]
\textbf{本专业:机械设计制造及其自动化} \\[4pt]
\begin{itemize}
  \item 毕业设计:人体工程学自动调节座椅(优秀毕业设计)
\end{itemize}

% 专业实践经验
\section*{专业实践经验}

\noindent
\textbf{\sffamily 沃尔沃汽车(Volvo Cars)} \hfill
\textit{中国上海} \\
\textbf{2023年11月 – 至今} \\[6pt]
\textbf{\large AUTOSAR AP/CP架构开发与确定性通信中间件专家} \hfill
\textit{2023年11月 – 至今}

\textbf{跨域融合软件架构设计与实现:}
\begin{itemize}
  \item 基于AUTOSAR Adaptive Platform标准设计SPA3/GPA HI集成架构,满足实时性、确定性、安全性要求
  \item 开发确定性通信中间件组件,实现关键业务流的QoS保障和确定性传输
  \item 设计TSN网络流量调度策略,优化ADAS和智驾数据的实时传输性能
\end{itemize}

\textbf{AUTOSAR工具链配置与代码生成优化:}
\begin{itemize}
  \item 创建yaml2arxml工具替代Davinci Developer,实现AUTOSAR配置文件的高效转换
  \item 开发stakeholder\_build\_template工具,管理SWC配置并自动生成ECUextract和.c/.h框架代码
  \item 负责AUTOSAR AP/CP工具链维护,确保代码生成、编译、部署流程高效可靠
  \item 实施ECUExtract\_Unflattened.arxml自动检查,解决Davinci Developer和Configurator配置问题
\end{itemize}

\textbf{车载以太网及通信协议开发:}
\begin{itemize}
  \item 负责SOME/IP服务在Davinci Configurator中的配置、部署和测试验证
  \item 使用CANoe配合VN5650设备进行车载以太网通信协议测试和性能分析
  \item 通过PuTTy访问Linux系统,使用劳德巴赫进行确定性通信组件的断点调试
  \item 实现DDS协议栈集成,支持跨域融合的高效数据交换
\end{itemize}

\vspace{8pt}

\noindent
\textbf{\sffamily 科世达(上海)机电有限公司,KOSTAL (Shanghai) Mechatronic Co., Ltd.} \hfill
\textit{中国上海} \\
\textbf{2020年12月 – 2023年11月}

\vspace{6pt}
\textbf{\large 嵌入式软件架构师 | 车身域控制器CEM专家} \hfill
\textit{2020年12月 – 2023年11月}

\textbf{确定性通信架构设计与实现:}
\begin{itemize}
  \item 基于AUTOSAR CP架构设计车身域控制器确定性通信方案,满足实时性和安全性要求
  \item 搭建集成测试软硬件环境,实现确定性调度算法验证和性能测试
  \item 设计产线终检程序和EMC测试软件,确保通信组件的可靠性
\end{itemize}

\textbf{Vector工具链深度应用与优化:}
\begin{itemize}
  \item 利用Python基于客户CAN/LIN/IO Matrix生成CANoe CAPL脚本和vTESTstudio配置文件
  \item 构建自动化集成测试平台,实现通信协议的自动化验证和回归测试
  \item 定位集成测试和系统测试中的通信问题,推动相关工程师完善确定性通信软件
\end{itemize}

\textbf{跨域融合项目经验:}
\begin{itemize}
  \item \textbf{宇通客车} - KBCM车身控制器项目 | 内灯模块通信架构设计 \hfill \textit{2022年8月 – 2023年8月}
  \item \textbf{长城汽车} - 欧拉好猫系列KBCM项目 | 确定性通信系统集成 \hfill \textit{2022年2月 – 2023年8月}
  \item \textbf{理想汽车} - 车和家系列KBCM项目 | 跨域通信架构负责人 \hfill \textit{2022年3月 – 2022年10月}
  \item \textbf{长城汽车} - 欧拉黑猫系列KBCM项目 | BLE/TBOX/VCU/ESCL通信集成 \hfill \textit{2021年1月 – 2021年12月}
\end{itemize}

% 技术项目经历
\section*{技术项目经历}

\noindent
\textbf{\sffamily 硕士论文 | 人体工学研究与商业组织研究所 (KIT)} \hfill
\textit{2019年8月 – 2020年8月}
\textbf{动态眼动追踪系统数据质量优化分析方法的设计与验证}
\begin{itemize}
  \item 设计实时数据采集系统,收集眼动追踪系统的点云追踪图像数据
  \item 使用开源目标识别AI模型Mask\_RCNN实现确定性数据处理算法
  \item 在Python中利用误差空间插值模型实现实时性优化,提升系统确定性性能
\end{itemize}

\vspace{8pt}

\noindent
\textbf{\sffamily 系统控制实践 | 系统控制研究所 (KIT)} \hfill
\textit{2018年10月 – 2019年1月}
\textbf{确定性控制系统设计与实现}
\begin{itemize}
  \item \textbf{倒立摆确定性控制:}使用状态空间方法建立确定性系统模型,在MATLAB/SIMULINK中实现实时控制算法
  \item \textbf{后轴试验台确定性调度:}设计确定性调度算法,实现系统极点补偿和解耦控制,通过传递函数优化PID控制器参数
\end{itemize}

% 专业技能
\section*{专业技能}

\textbf{AUTOSAR架构与工具链:}
\begin{itemize}
  \item \textbf{AUTOSAR AP/CP:} Adaptive Platform架构、ARA服务、配置管理、代码生成
  \item \textbf{Vector工具链:} DaVinci Developer\&Configurator、CANoe、vTESTstudio、ECUExtract
  \item \textbf{确定性通信:} SOME/IP、DDS、CAN FD、车载以太网、TSN协议族
\end{itemize}

\textbf{编程语言与开发工具:}
\begin{itemize}
  \item \textbf{编程语言:} C++11/14/17、Python、Java、Groovy、MATLAB(Stateflow)
  \item \textbf{系统架构:} 面向对象设计、设计模式、多线程编程、确定性调度算法
  \item \textbf{调试工具:} iSYSTEM winIDEA、Lauterbach Trace32、CANoe、劳德巴赫
\end{itemize}

\textbf{通信协议与标准:}
\begin{itemize}
  \item \textbf{车载通信:} SOME/IP、DDS、CAN/LIN、车载以太网、TSN协议
  \item \textbf{系统标准:} POSIX系统、实时性优化、ISO 26262功能安全、ASPICE
\end{itemize}

\textbf{外语能力:}
\begin{itemize}
  \item 英语(雅思:6.5)- 能够阅读英文技术文档和进行技术交流
  \item 德语(TestDaf:16/C1)- 具备德语工作环境适应能力
\end{itemize}

% 特别经历和荣誉
\section*{特别经历和荣誉}

\begin{itemize}
  \item \textbf{科世达亚洲总部AE电子开发部年会主持人} \hfill \textit{2023年2月16日}
  \item \textbf{科世达(上海)机电有限公司2022年度优秀员工} \hfill \textit{2022年1月 – 2022年12月}
  \item \textbf{科世达亚洲总部AE电子开发部技术表彰大会主持人} \hfill \textit{2022年2月28日}
  \item \textbf{上海市防疫封控期间驻守公司,确保关键项目按时交付} \hfill \textit{2022年3月 – 2022年6月}
\end{itemize}

\end{document}