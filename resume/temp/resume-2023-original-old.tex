\documentclass[11pt,a4paper]{article}
\usepackage{ctex}
\usepackage{geometry}
\geometry{left=2cm,right=2cm,top=2cm,bottom=2cm}
\usepackage{titlesec}
\usepackage{enumitem}
\usepackage{fancyhdr}
\usepackage{hyperref}

% 页面设置
\pagestyle{fancy}
\fancyhf{}
\renewcommand{\headrulewidth}{0pt}
\renewcommand{\footrulewidth}{0pt}

% 标题格式
\titleformat{\section}
  {\Large\bfseries\sffamily\raggedright}
  {}
  {0em}
  {}
  [{\titlerule[0.5pt]}]

\titleformat{\subsection}
  {\large\bfseries\sffamily}
  {}
  {0em}
  {}

\titlespacing{\section}{0pt}{10pt}{6pt}
\titlespacing{\subsection}{0pt}{6pt}{4pt}

% 列表格式
\setlist{
  leftmargin=0pt,
  itemsep=3pt,
  parsep=0pt,
  topsep=3pt,
  labelsep=8pt
}

% 超链接设置
\hypersetup{
  colorlinks=true,
  urlcolor=blue,
  linkcolor=black
}

\begin{document}

% 个人信息
\begin{center}
  {\Huge\bfseries\sffamily 张凯 (Kai Zhang)}\\[6pt]
  {\large\texttt{Tel.: +86 15122986177} | \texttt{E-Mail: 15122986177@163.com}}\\[6pt]
  {11.03.1992}
\end{center}

\vspace{12pt}

\hrule
\vspace{12pt}

% 教育背景
\section*{教育背景}

\noindent
\textbf{\sffamily 卡尔斯鲁厄理工学院 (KIT)} \hfill
\textit{Karlsruhe Institut für Technologie}\\
\textbf{机电一体化与信息技术 硕士} \hfill
\textit{2017年4月 – 2020年10月}\\[4pt]
\begin{itemize}
  \item 专业方向1:工业自动化
  \item 专业方向2:机器人技术
\end{itemize}

\vspace{8pt}

\noindent
\textbf{\sffamily 河北工业大学} \hfill
\textit{中国天津}\\
\textbf{机械设计制造及其自动化 学士} \hfill
\textit{2012年9月 – 2016年7月}\\[4pt]
\begin{itemize}
  \item 毕业论文:人性化自动调节椅的设计
\end{itemize}

% 工作经历
\section*{工作经历}

\noindent
\textbf{\sffamily KOSTAL (Shanghai) Mechatronic Co., Ltd.} \hfill
\textit{集成开发负责人/软件工程师}

\vspace{6pt}
\textbf{CEM - 车身域控制器项目 | 集成开发负责人} \hfill
\textit{2020年12月 – 2022年12月}
\begin{itemize}
  \item 搭建集成测试的硬件和软件环境
  \item 编译软件、烧录硬件并进行集成测试
  \item 设计生产线最终检验程序
  \item 设计EMC上位机和PCBA的EMC测试所需专用测试软件
  \item 定位集成测试和系统测试中的问题,并与相关工程师沟通改进软件
  \item 定位客户问题并帮助相关工程师改进软件
  \item 向客户发布软件并提供维护服务
\end{itemize}

\vspace{6pt}
\textbf{Vector工具链的车身控制器自动化集成测试平台 | 创建者、推广者} \hfill
\textit{2023年1月 – 2023年6月}
\begin{itemize}
  \item 基于客户的CAN/LIN/IO矩阵使用Python生成CANoe CAPL脚本和vTESTstudio配置文件
  \item 在vTESTstudio中构建自动化测试项目
  \item 根据客户功能规格编写自动化测试用例
  \item 将自动化测试用例导入CANoe进行自动化测试
\end{itemize}

\vspace{6pt}
\textbf{客户项目经验}
\begin{itemize}
  \item \textbf{宇通客车} - KBCM(车身控制器)项目 | 内饰灯模块软件开发工程师 \hfill \textit{2022年8月 – 2023年8月}
  \item \textbf{长城汽车} - 欧拉好猫系列KBCM项目 | 集成开发负责人 \hfill \textit{2022年2月 – 2023年8月}
  \item \textbf{江西五十铃} - PEPS(无钥匙进入系统) \& ESCL(电子转向锁)项目 | 集成开发负责人 \hfill \textit{2022年3月 – 2023年3月}
  \item \textbf{理想汽车} - 车和家系列KBCM项目 | 集成开发负责人 \hfill \textit{2022年3月 – 2022年10月}
  \item \textbf{长城汽车} - 欧拉黑猫系列KBCM项目 | 集成开发负责人 | BLE/TBOX/VCU/ESCL认证模块软件工程师 \hfill \textit{2021年1月 – 2021年12月}
\end{itemize}

% 学术项目经历
\section*{学术项目经历}

\noindent
\textbf{\sffamily 硕士论文 | 人体工学研究与商业组织研究所 (KIT)} \hfill
\textit{2019年8月 – 2020年8月}
\textbf{动态眼动追踪系统数据质量优化分析方法的设计与验证}
\begin{itemize}
  \item 设计实验收集眼动追踪系统的点云追踪图像数据
  \item 使用开源目标识别AI模型Mask\_RCNN将图像数据转换为坐标数据集
  \item 在Python中使用误差空间插值模型补偿眼动追踪系统,实现优化眼动追踪系统数据质量的目标
\end{itemize}

\vspace{8pt}

\noindent
\textbf{\sffamily 系统控制实践 | 系统控制研究所 (KIT)} \hfill
\textit{2018年10月 – 2019年1月}
\textbf{倒立摆控制与后桥试验台控制}
\begin{itemize}
  \item \textbf{倒立摆控制:}使用状态空间方法建立系统模型,在MATLAB/SIMULINK中仿真系统控制,设计观测器并用最小二乘控制器实现平衡控制
  \item \textbf{后桥试验台控制:}使用状态空间方法建立系统模型,进行系统极点补偿和解耦控制,通过传递函数计算PID控制器参数,在MATLAB/SIMULINK中仿真并调整PID参数
\end{itemize}

\vspace{8pt}

\noindent
\textbf{\sffamily 本科毕业论文 | 河北工业大学} \hfill
\textit{2016年1月 – 2016年6月}
\textbf{人性化自动调节椅的设计}
\begin{itemize}
  \item 研究人性化座椅结构设计
  \item 使用相关机构设计椅子调节方式
  \item 在SolidWorks中按1:1比例构建椅子3D模型
  \item 在AutoCAD中制作2D装配图和2D零件图
\end{itemize}

\vspace{8pt}

\noindent
\textbf{\sffamily 其他学术项目}
\begin{itemize}
  \item \textbf{点焊机器人机构设计} (2015年2月 – 2015年4月):结合相关机构设计焊接机器人各悬臂几何结构,在Adams中组装点焊机器人,在Adams中仿真分析动态性能
  \item \textbf{PLC导轨滑块生产线设计} (2015年5月 – 2015年7月):根据负载要求和选定结构计算所需导轨滑块的功率和负载,选择西门子导轨滑块,将制造商3D模型导入SolidWorks进行虚拟模型装配,使用Adams对关键节点进行力学分析,设计满足装配线要求的PLC程序
\end{itemize}

% 专业技能
\section*{专业技能}

\begin{itemize}
  \item \textbf{Vector工具链:}DaVinci Developer \& Configurator、CANoe、vTESTstudio
  \item \textbf{编程语言:}Python、C++、MATLAB(Stateflow)
  \item \textbf{代码工具:}Smart SVN、SourceInsight、Eclipse
  \item \textbf{调试工具:}iSYSTEM winIDEA
  \item \textbf{效率工具:}Xmind
\end{itemize}

% 语言能力
\section*{语言能力}

\begin{itemize}
  \item \textbf{普通话:}母语
  \item \textbf{英语:}雅思6.5分
  \item \textbf{德语:}TestDaf 16分/C1水平
\end{itemize}

% 特殊经历与荣誉
\section*{特殊经历与荣誉}

\begin{itemize}
  \item \textbf{2023年2月16日:}主持科世亚亚洲总部AE开发部中国新年晚宴
  \item \textbf{2022年1月 – 2022年12月:}获得科世达(上海)机电有限公司2022年度优秀员工奖
  \item \textbf{2022年2月28日:}主持科世亚亚洲总部AE开发部年度表彰大会
  \item \textbf{2022年3月 – 2022年6月:}上海疫情封控期间驻公司协助完成多项紧急任务
\end{itemize}

\end{document}